% !TeX root = ../main.tex
\section{Future work}

\subsection{User-tests}
As per the conclusion of this report, the HCL language has not been usability-tested.
The target-group of HCL consists of two sub-groups, novices with no programming experience, and novices with limited programming experience.

To conduct the usability-test, the sub-group with no programming experience, would have to be taught simple programming constructs in HCL and general programming paradigms.
The sub-group with limited programming experience, would be asked to write simple programs in a language they know, and then write the equivalent programs in HCL.

The first sub-group would be used to determine whether HCL is intuitive.
The second sub-group would be used to evaluate whether HCL simplifies the programming paradigms enough.

\subsection{Improvements to the language}
While HCL is a complete language, there are still improvements that could be made.

\textbf{Working Arduino}\\
While the project group wanted HCL to run on the Arduino platform, this feature was not completed in time for the project hand-in.
The Arduino does not allow lambdas, or the import of the library that is used in the C++ portion of the code.
A solution could be to convert all lambdas to full functions and reference these instead.
Another option could be to write a lambda type for the Arduino.

\textbf{Object oriented features}\\
The object oriented paradigm, as seen in Java and C\# are often taught to novices.
To allow novices to easier transition to these languages, object oriented features could be beneficial for HCL

\textbf{Interpreter}\\
An interpreter for HCL could be beneficial, at it could allow novices to quickly try out syntax, using the command-line, instead of fully compiled programs.

\textbf{Dictionaries}\\
Dictionaries were deemed NTH during the development process.
It would benefit HCL to have a powerful data-structure, such as dictionaries.
This would make it simpler to implement complex algorithms in HCL, for example sorting algorithms and numerical counting algorithms.
It would also simplify the process of optimizing naive implementions, especially optimization by dynamic programming.

