% !TeX root = ../main.tex
\section{Future work}
\label{sec:futureWorks}

\subsection{User-tests}
As per the conclusion of this report, the HCL language has not been usability-tested.
The target-group of HCL consists of two sub-groups, novices with no programming experience, and novices with limited programming experience.

To conduct the usability-test, the sub-group with no programming experience, would have to be taught simple programming constructs in HCL and general programming paradigms.
The sub-group with limited programming experience, would be asked to write simple programs in a language they know, and then write the equivalent programs in HCL.

The first sub-group would be used to determine whether HCL is intuitive.
The second sub-group would be used to evaluate whether HCL simplifies the programming paradigms enough.

\subsection{Improvements to the language}
While HCL is a complete language, there are still improvements that could be made.

\textbf{Working Arduino}\\
While the project group wanted HCL to run on the Arduino platform, this feature was not completed in time for the project hand-in.
The Arduino does support lambdas, but not the specific type used for HCL, the \textit{\textbf{std::function type}} from the "functional" \texttt{C++} standard library.
A solution could be to write a custom lambda type, specifically for the Arduino.

\textbf{Interpreter}\\
An interpreter for HCL could be beneficial, as it could allow novices to quickly try out syntax, using the command-line, instead of fully compiled programs.

\textbf{Dictionaries}\\
Dictionaries were deemed NTH during the development process.
It would benefit HCL to have a powerful data-structure, such as dictionaries.
This would make it simpler to implement complex algorithms in HCL, for example sorting algorithms and numerical counting algorithms.
It would also simplify the process of optimizing naive implementions, especially optimization by dynamic programming.

\textbf{Hash collisions}\\
As mentioned in section \ref{sec:gencplusplus}, \textit{generated \texttt{C++} code}, HCL can potentially generate hash code collisions, when naming types and variables in \texttt{C++}.

There are a couple of potential solutions to this problem.
One solution would be to only do hashing on individual illegal characters, thus retaining some of the names defined in HCL.

Another solution could be to implement a list of valid names for illegal characters.
This would mean that \textbf{+} would become \textbf{plus}, \textbf{-} would become minus, and so forth.
The issue here, would be if the user created a function with those names.

Yet another possible solution is to map all characters to the corresponding UTF-8 character code value and join them to an identifier (the down side to this solution, is that it will render large identifier sizes in the generated \texttt{C++} code).

\subsection{Bugs in Compiler}
There are a few known bugs currently found in the compiler at the time of writing this report. 
The following is a run-down of the known bugs and their potential complications in the program. 

\textbf{Overloading in High-Order Functions}\\
Although both overloading and high-order functionality works separately, once both techniques are used together the compiler faults in determining the correct result.
The problem potentially occurs, when an overloaded function is passed as an argument for another function. 
If the first function the compiler recognizes does not use the correct input-parameters, the compiler throws an exception.
However, if the first function found happens to be the correct one, the compiler sees no faults and works expectantly.
The problem is shown in snippet \ref{lis:highover}, where the first overloaded function declared is of wrong parameter-type, and therefore causes a compiler exception. 
\begin{lstlisting}[language=Kotlin,label=lis:highover,caption=Overloading in high-order functions problem.]
var f = (func[num, num] n) : num { 2 n }
var p = (txt i): txt { "hello" }
var p = (num i): num { 5 * i }
:p f 	#should print 10, but fails. 
\end{lstlisting}

\textbf{None Returns}\\
The following is more of an oversight than a bug. 
The problem occurs when a function, that is signified to not return anything, contains a return statement. 
The compiler does not recognize the return statement, and will thus not throw an exception. 
However, the oversight might lead to confusion or prolonged debugging-time. 
The problem is shown in snippet \ref{lis:nonereturn}.
\begin{lstlisting}[language=Kotlin,label=lis:nonereturn,caption=Returning from a none-function.]
func f = (): none { 
"something to print" print
return 5 
} 
#Causes no problems, but should.
\end{lstlisting}


