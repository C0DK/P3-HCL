% !TeX root = ../main.tex
\chapter{Discussion}
\section{Discussion}
This chapter reflects upon the choices made in regards to the developed solution and report, as well as how the group delegated work to the individual group members and assured the quality of said work.
Furthermore, the major choices, not already touched upon, are discussed in this chapter.

The goal of HCL is to develop an "English-like" language.
The choice of an English-like language, instead of a Danish-like language, is to easier introduce programming concepts to the target-group.
By being an English-like language, HCL expressions are designed to naturally resemble english written sentences.
If proper naming conventions are followed, the written code becomes more intuitive.
As an example, snippet \ref{lis:discussionintuitive} shows why this is the case.

\begin{lstlisting}[language=HCL,label=lis:discussionintuitive,firstnumber=1,caption=Example of the intuitive nature of HCL]
func doSomething = () : none { Some Code that does stuff }
func thisIsTrue = () : bool { Some code that returns a bool }

:doSomething while :thisIsTrue
\end{lstlisting}

As discussed in the problem analysis, the majority of the target group has no understanding of operators.
Users of HCL do not need to know the basics of logic operations as presented in discrete mathematics.
This is because the implicit understanding of logical constructs, is natural in HCL, as the built-in operators have english names, instead of just being a character.
As is evident by the statements made by the gymnasium students, novices would not be able to understand the logic operation shown below.
\begin{center}
	$(P \wedge Q) \vee R$
\end{center}
Naturally, because they would not be familiar with the precedence or association of the logical operators.
In HCL this becomes much simpler, since all function calls are left associative and no operators exists.
The snippet \ref{lis:discussionoperatorIntuitiveness} shows the equivalent HCL code.
\begin{lstlisting}[language=HCL,label=lis:discussionoperatorIntuitiveness,firstnumber=1,caption=Example of the intuitive nature of HCL]
(P and Q) or R
\end{lstlisting}
Types in HCL are immutable, while variables are mutable.

This forces novices to think upon their naming, and further consider what variables are already in use.

As an example, consider a list.
When a list is created, it is an empty list.
When a value is then appended to the list, it is not the same list as previously, but instead a new list containing the new value.
When yet another value is appended to the list, it is again a new list containing both of the appended values.

The variable the list is assigned to, cannot be reassigned to another type, say a string.

\subsection{Handwritten lexer/parser}
The group made the decision early on in the project, to write the entire compiler by hand.
The choice to do this was founded on the belief, that by writing the entire compiler by hand, and not using a lexer/parser generator tool, the group would be more knowledgeable regarding the developed solution.

While this is most likely true, the disadvantage of this approach, was that the lexer and parser were written in tandem with the semantics.
By using a generator tool, the project group would have been forced to fully define the semantics before beginning on the lexer and parser.
This would have meant that some issues that were only spotted late in the project pipeline, would have been spotted earlier.
However, the group still believe that the correct choice was to write the whole compiler by hand, as some problems have been solved with unique solutions.

\subsection{Mini-computers}
The project has been, primarily, developed with mini-computers in mind.
In particular, the Arduino platform has been targeted.
This is because the Arduino Uno is often used in education, and it makes sense to develop for platforms that are already readily available and in use.

However, as the project group found out, the Arduino platform does not allow for import of standard \textbf{C++} library functions.
This means that the group ended up instead focusing on getting HCL to run on desktop.
As mentioned in section \ref{sec:cplusplusfunctions}, a custom implementation of lambdas is required, for HCL to be usable on the Arduino platform.
This is saved for future work.

HCL does however work on the NodeMCU, which is an Arduino clone with a more powerful processor, the tensilica core.
The NodeMCU allows for import of \textbf{C++} libraries, due to the xtensa-toolchain\footnote{https://github.com/noduino/xtensa-toolchain}, which allows desktop programs to be ported over in a rather straightforward manner.

\section{Quality Assurance}
During the development of the HCL compiler, quality gates played a large part.

The project was developed with everything, including the report, hosted on the git\cite{GitProtocol} version control system, utilizing the free Github implementation of it.
The group used branching for features, and used CI\footnote{Continuous Integration; automatically running tests upon updates} to make sure that new features did not break unit tests nor code standards.
This was also done by not allowing the merging of branches into the master branch, without having a high enough code coverage of unit tests and that all tests succeed.

On top of CI, the group also had manual review of report and code, to make sure both was of adequate quality. 
Each feature needing a review by at least one other person in the group before entering a release branch.
To make sure that everyone was on top of development and changes, features were merged into a weekly release branch, which was then reviewed by everyone in the group at the end of the week and then merged into the master branch.

To create an even higher level of transparency, weekly status meetings were held following release branches.
These status meetings discussed the previous week, as well as the issues to be address in the coming week. 
A large monthly status meeting also took place, to look at progress on a larger scale. 

Jira\cite{Jira}, a project management system, was used to address issues and progress.
Using integrations for git, made issues automatically update based on names or keywords in commit messages and branches.

The group tried to follow a preset milestone plan, however it was shifted a bit over the course of the development process.

\subsection{Automatic tests}

As a compiler generates other programs, it is important to make sure that the compiler, and the programs generated with it, works exactly as expected.
Therefore unit tests and integration tests were deemed quite important, and a high coverage, both of lines as well as branches, was deemed even more important than previous projects done by the group. 

When developing a new module of the compiler, unit tests was created in parallel, testing each subset of features to make sure that they work in isolation. 
Which made debugging easier as development progressed.

Unit tests was created with the JUnit\cite{JUnit} framework, a Java Unit testing framework, and the Spek\cite{SpekFramework} framework, a specification testing framework. 
The group heard of Spek halfway through the development process and it ended up being the preferred testing framework, however not all existing tests were ported to Spek, as these work conjointly.

Specification tests are based on the principle of writing, in English, what a test should do accompanied by code executing the relevant actions.
This is shown in snippet \ref{lis:specExample}.

\begin{lstlisting}[language=Kotlin,label={lis:specExample},caption={An sample unit tests with spec}]
class SimpleTest : Spek({
	describe("a calculator") {
		val calculator = SampleCalculator()
		
		it("should return the result of adding the first number to the second number") {
			val sum = calculator.sum(2, 4)
			assertEquals(6, sum)
		}
	}
})
\end{lstlisting}

System tests, testing of the whole pipeline, was also valued highly.
This was utilized as a way of making sure that a given snippet of HCL code should create a program that would print the correct text and/or return the correct number.
To do this a test was developed that takes a list of files and checks whether each file has the expected output.
This was done by having a comment in the file, that includes the expected output of the program, which the testing framework would then process. 

Such a script can be seen in snippet \ref{lis:testHelloWorld}, where the first line is the comment that tells the testing framework that the program should in fact print "Hello World!", whereas the second line prints "Hello World!" in the HCL language.
\begin{lstlisting}[language=HCL,label=lis:testHelloWorld,firstnumber=1,caption={A HCL test-script as part of the testing framework}]
# This should print Hello World!
"Hello World!" print
\end{lstlisting}

The script above is then executed by the testing function which is, in a shortened version, portrayed in snippet \ref{lis:testHclPrograms}
\begin{lstlisting}[language=Kotlin,label={lis:testHclPrograms},caption={Shortened version of the test for system tests}]
object TestHclPrograms : Spek({
	val files = listOf(
		"HelloWorld.hcl",
		...
		"while.hcl",
	)
	files.filter { it.endsWith(".hcl") }.forEach { file ->
		given(file) {
			val fileContent = javaClass.classLoader.getResource(file).readText()
			val constraints = fileContent.split("\n").first().split("should 		").drop(1).map { it.split(" ") }
			val expectedReturn = constraints.firstOrNull { it.first() == "return" }?.get(1)?.toInt() ?: 0
			val expectedPrint = constraints.firstOrNull { it.first() == "print" }?.drop(1)?.joinToString(" ") ?: " "
			
			val outputFiles = generateFilesFromCode(fileContent)
			val output = compileAndExecuteCpp(outputFiles, file.split(".").first(), keepFiles)!!
			assertEquals(expectedReturn, output.returnValue,
			"expected RETURN_CODE=$expectedReturn, was ${output.returnValue}\n" +
			"full output:\n${output.string}")
			assertEquals(expectedPrint, output.string)	
		}
	}
})
\end{lstlisting}

Line 3 to 5, in snippet \ref{lis:testHclPrograms}, is the list of all the scripts, currently 10 in the non shortened version. 
Line 9 to 12 handles processing of the expected outcome, whereas line 14 and 15 are compilation of the program, and the remaining lines are for checking the result of the test

\subsection{Quality Assurance retrospect}

These different software solutions and principles, combined with a sprint-based SCRUM-like development process, created a streamlined development process which helped make a lot of the features possible within the time frames of this project.
