% !TeX root = ../main.tex
\chapter{Conclusion}
\section{Conclusion}
The goal of this project has been to create a compiler for a programming language.
The resulting solution is the programming language HCL and the HCL compiler.

\subsection{Conclusion of the problem-statement}
HCL is designed with novices in mind, while also implementing high-order functionality, with intuitive syntax and semantics.
HCL is a hybrid language of the functional and imperative paradigms, with dynamic binding of values.
HCL has English-like syntax, that also implements high-order functionality.
%However, as of the conclusion of this report, no user-tests have been conducted (This sentence is really fucking out of place. I have removed it.)

The similarities with the English language is well exemplified in snippet \ref{lis:ConclusionNicolajEksempel}.
%Decide and rewrite.
\begin{lstlisting}[language=hcl,label=lis:ConclusionNicolajEksempel,caption=English-like syntax example]
1 to 5 map { value squared } print
\end{lstlisting}
In snippet \ref{lis:ConclusionNicolajEksempel}, the values from 1 to 5 are mapped with the value squared and then printed.
The code itself is pretty self-explanatory, and matches the English explanation a lot.
Having the first parameter on the left-hand side of the function call, allows for writing code, that reads like English.

Since user-tests have not been conducted, it cannot be verified whether the English-like syntax allows for novices to have an easier transition from English to HCL.

The project group seeks to improve the, otherwise steep, learning curve of high-order functionality.
More precisely, HCL's handling of lambda expressions need to be intuitive.
This was accomplished by syntactically treating any lambda expression as an anonymous implicit function definition.
Snippet \ref{lis:ConclusionMapFilter} shows an example of how the passing of a lambda expression look in HCL.

\begin{lstlisting}[language=hcl,label=lis:ConclusionMapFilter,caption=High-order functionality in HCL]
[2, 4, 6, 8] map { value * 2 } where { value greaterThan 10 } print
\end{lstlisting}

As mentioned, user-tests have not been conducted.
Therefore, it cannot be verified whether the high-order functionality of HCL would be considered intuitive by novices.

The programming paradigms have been simplified in HCL.
Every operator in HCL is a function, and every function call looks like an operator operation in most other languages (2 + 2, 2 * 2).
Since function calls behave in the same way as operators, and accept input parameters in the same manner, the syntax is uniform, and without ambiguity when dealing with built-in and user defined functions.
Snippet \ref{lis:ConclusionSimplified} shows an example of how this concept is handled in HCL.

\begin{lstlisting}[language=hcl,label=lis:ConclusionSimplified,caption=Built-in and user defined function]
var plus = (num left, num right): num { left + right }
#built in addition
var res1 = 2 + 3
#user defined addition
var res2 = 2 plus 3 
\end{lstlisting}

The two function calls on lines 3 and 5 both evaluate to the same value and with no ambiguity in the syntax.

\subsection{Conclusion on the solution}
HCL is a fully fledged, Turing complete, language.
Every part of the syntax has been semantically defined. 
The compiler is capable of compiling most syntactical constructs that are valid in HCL to \texttt{C++} , barring any bugs and faults in the compiler as of the writing of this report.

The HCL compiler comprises three modules, each equivalent to a stage in the compilation process.
First, a lexer, which converts text to a stream of tokens
Second, an LL(K), recursive decent, single pass parse with typechecking, which builds an AST.
Third, a codegenerator which traverses the AST and generates \texttt{C++} code.

All parts of the HCL compiler is written by hand.
A unique aspect of the HCL compiler is that the type-checking is done in the parser, so that the symbol table does not need to be saved for the code-generator.
This makes the HCL compiler a bit more efficient.

Since the Arduino Uno does not include a default lambda type, the HCL language cannot run on the Arduino platform as of this conclusion.
It does however work on the ESP8266, which is a microcomputer with equivalent specs to the Arduino Uno.
To get HCL to work on the Arduino UNO, a custom lambda data structure could be implemented.
This will be discussed further in section \ref{sec:futureWorks}.

While the language has not been tested with the target group, both the compiler and the language has been well tested by the project group.
The lexer, parser and code generator are all tested, with a code coverage of 88\% achieved through 298 individual tests.
The compiler, as a whole, is also well tested, with correct HCL source code.
The HCL language and the HCL compiler are both ready for usability-testing with the target group.



