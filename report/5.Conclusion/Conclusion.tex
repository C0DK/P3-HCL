% !TeX root = ../main.tex
\chapter{Conclusion}
The goal of this project was to create a compiler for a programming language of the groups own design.
The resulting solution is the programming language HCL and the HCL compile.
HCL is designed with novices in mind, while also implementing high-order functionality, with intuitive syntax and semantics.
HCL is a functional language, with dynamic binding of values.
HCL has English-like syntax, that also implements high-order functionality, however, as of of the conclusion of this report, no user-tests have been conducted.

The similarities with the English language is well exemplified in snippet \ref{lis:ConclusionWhile}.
\begin{lstlisting}[language=hcl,label=lis:ConclusionWhile,caption=English-like syntax example]
var j = 3
func subtract = () : none { j = j - 1 }
func jIsGreaterThan1 = () : bool { 1 lessThan j }
:subtract while :jIsGreaterThan1
\end{lstlisting}
In snippet \ref{lis:ConclusionWhile}, lines 1 to 3 consists of declarations, line 8 consists of the function-call to While.
The While function takes two Lambda expressions as arguments, in this case, two function-calls.
The similarities between the function-call on line 8 and the equivalent English sentence is obvious.

Since user-tests have not been conducted, it cannot be verified whether the English-like syntax allows for easier transition, from English to HCL, for novices.

The project group seeks to improve the, otherwise steep, learning curve of high-order functionality.
More precisely, HCL's handling of lambda expressions need to be intuitive.
This was accomplished by syntactically treating any lambda expression as an anonymous implicit function definition.
Snippet \ref{lis:ConclusionMapFilter} shows an example of how the passing of a lambda expression look in HCL.

\begin{lstlisting}[language=hcl,label=lis:ConclusionMapFilter,caption=High-order functionality in HCL]
[2, 4, 6, 8] map { value * 2 } filter { value greaterThan 10 } print
\end{lstlisting}

Since user-tests have not been conducted, it cannot be verified whether the high-order functionality of HCL would be considered intuitive by novices.

The programming paradigms have been simplified in HCL.
Every operator in HCL is a function, which results in the syntax being uniform in every aspect of the language.
This simplification removes ambiguity in the syntax when dealing with built-in and user defined functions in the same context.
Snippet \ref{lis:ConclusionSimplified} shows an example of how this concept is handled in HCL.

\begin{lstlisting}[language=hcl,label=lis:ConclusionSimplified,caption=Built-in and user defined function]
var plus = (num left, num right): num { left + right }
#built in addition
var res1 = 2 + 3
#user defined addition
var res2 = 2 plus 3 
\end{lstlisting}

The two function calls on lines 3 and 5 both evaluates to the same value and with no ambiguity in the syntax.

This finishes the conclusion of the problem-statement.

HCL is a fully fledged, Turing complete language.
Every part of the syntax has an equivalent semantic context, and the compiler is capable of compiling any HCL valid syntactical construct to C++.

The HCL compiler is an LL(K), recursive decent, single parse, compiler.
All parts of the HCL compiler is written by hand.
A unique aspect of the HCL compiler is that the type-checking is done in the parser, so that the symbol table does not need to be saved for the code-generator.
This makes the HCL compiler a bit more efficient.

Since the Arduino does not include a default lambda type, the HCL language can not run on the Arduino platform as of this conclusion.

While the language has not been tested with the target group, both the compiler and the language has been well tested by the project group.
The lexer, parser and code generator are all tested, with good coverage, the compiler as a whole is also well tested with valid HCL source code.
The HCL language and the HCL compiler are both ready for usability-testing with the target group.

%The development of a language, and a compiler for that language, is a complex exercise, and HCL is not different.
%From the project supervisor a complete language, being a language that is actually usable\footnote{Complete programs can be written} and is turing complete, was not required.
%It was decided by the project group that HCL would be a complete language. 

%During development a "test-driven" method was employed.
%This meant that the entirety of the compiler, from the lexer to the code generation, with complete integration tests is completely tested.
%This means that the individual parts of the compiler


