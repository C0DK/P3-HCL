% How to do quick maths tutorial (HD).
% The lines in the derivations are just fraction lines (made with \cfrac{}{}).
% Sideways T (turnstile - bom in Danish) is \vdash.
% The weird, tall "<" ">" brackets are \langle and \rangle.

\section{Semantics of HCL}
This section concerns with the semantics of HCL.
For the sake of easing the reading of this section, semicolons used in semantic rules are to be left associative.
For example; $S_1;S_2;S_3$ must be read as $S_1;(S_2;S_3)$.
The components of this statement are $S_1$ and $S_2;S_3$.
The notation used to formalize the semantics of HCL in the following section will be operational structural semantics.
All build-in functionality of HCL will have a defined semantic at the end of this section.
All transitions are written using Big-Step semantics, BSS for short.
The Environment-store-model will be used to describe how the state of variables and procedures are bound.
HCL allows for the use of ';' and '{\}n' for line endings.
for simplicity, will the abstract syntax and rules in this section utilize ';' for line endings. 

\subsection{Scoping of HCL}
The scope rules of a language determines which variable environment and procedure environment are being accessed, when using the name of a variable or function name.
Since the semantics of HCL will be stated using operational structural semantics, the scope rule can be shown by simple accessing the environment required by the scope rules.
The scope rules of HCL are fully static.
This means that the bindings accessed when using variable or functions names, will be the ones that were bound when the variable or function was declared.
The reader should note that, since HCL is utilizing fully static scope rules, transitions for function calls needs special components to successfully enable recursive calling of function.
In other words, when make a recursive call in HCL the bindings of the procedure environment will be dynamic.

\subsection{Abstract Syntax}
The abstract syntax will now be presented.
For convenience, the following notational conventions will be followed throughout this section.

In the rest of the section $t1, .
.
.
 , tn$ will be abbreviated with $t\textasciitilde$ and $t1 .
 .
 .
  tn$ with $t^*$ ($t^+$ in the non-empty case).

The syntax categories and meta-variables used in the semantic can be seen in table \ref{tab:metaVarab}. 
\begin{center}
	\begin{table}[ht]
		\centering
		\caption{The syntax categories and meta-variables used in the operational semantics of HCL}
		\label{tab:metaVar}
		\begin{tabular}{ll}
			\textbf{Metavariable} & \textbf{Name} \\
			$a \in A_{exp}$ 	  & Arithmetic expression \\
			$n \in Num$			  & Number Literal \\
			$x \in Var$           & Variable \\
			$b \in B_{exp}$		  & Boolean expression\\ 
			$tf \in {True,False}$ & Boolean literal \\
			$t \in Txt_{exp}$     & Text expression\\
			$str \in {\Sigma^*}$  & String Literal \\
			$list \in list_{exp}$ & List expression \\
			$lst \in List$		  & List literal\\
			$tupple \in tpl_{exp}$& Tupple expression\\
			$tpl \in Tupple$	  & Tupple Literal\\
			$L \in Lambda$		  & Lambda expression\\
			$e \in Exp$			  & Expression\\
			$el \in Lit$		  & Expression Literal\\
			$p \in Pnames$        & Procedure names\\
			$L \in Lambda$        & Statement Lambdas\\
			$S \in Stm$           & Statement\\
			$f \in Fnames$ 		  & Function names\\
			$D_V \in Dcl_V$		  & Variable declaration \\
			
			$T \in Type$          & Types
		\end{tabular}
	\end{table}
\end{center}

For the rest of this section, it will be assumed that functions exists with domains equal to the sets of literals associated with the various expressions' syntax categories, and with co-domains equal to the values extracted from those literals.
The abstract syntax of HCL can be seen below.

$a ::= n | x | a_1+a_2 | a_1-a_2 | a_1*a_2 | a_1/a_2 | (a)$\\
$b ::= tf| x | a_1<a_2 | a_1>a_2 | a_1<=a_2| a_1>=a_2| a_1==a_2 | b_1\ and\ b_2 | b_1\ or\ b_2 | b\ not | (b)$\\
$t ::= str| x | t_1 + t_2 | t\ toString | (t)$\\
$list ::= lst | x | list\ @\ a | (list)$\\
$tupple ::= tpl | x | tupple\ @ | (tupple)$\\
$e ::= a | b | t | list | tupple | call_{exp}$\\
$eOrL ::= e | L$\\
%...
$Fcall ::= f | eOrL\ f\ eOrL^*$\\
%...
$L ::= \{\ S\ \} | \{\ S\ return\ e\ \} | \{\ S\ return\ S\ \} | \{\ S;\ e\ \}$\\
%...
$S ::= x = e | S_1;S_2 | b\ then\ L | L\ while\ b | list\ forEach\ L | p | eOrL\ p\ eOrL^* | D_V | D_P | \epsilon$\\
$D_v ::= T\ x\ =\ e;\ S | T\ x\ = call_{exp};\ S | T\ x;\ D_v | \epsilon$\\
$D_p ::= T\ pOrf\ =\ ([T\ x]\sim):\ T\ L;\ S |T\ pOrf;\ S | \epsilon$

\textbf{\large{Conventions}} \\
For each variable $x$, an arbitrary variable-environment $env_V$ describes which address $adr$, $x$ is bound to.
For simplicity, the set of all possible addresses, denoted $Adr$, is assumed to be equal to the set of all integers.
\begin{center}
	$Adr = \mathbb{Z}$
\end{center}
The set of all variable-environments, denoted \textbf{$Env_V$}, is the set of all partial functions from variables to addresses.
\begin{center}
	$Env_V = Var \cup \{next\} \rightharpoonup Adr$
\end{center}
For every $adr$, the function $new : Adr \rightarrow Adr$ returns the next $adr$, whether it is free or not.
\begin{center}
	$new (adr) = adr + 1$
\end{center}
A store $sto$ describes what values $v$ are pointed to by the addresses found in $Adr$.

With the variable bindings found in $env_V$ and the address content found in $sto$, the expressions are capable of returning the values of the $sto$.
This means that every transition of expressions must be on the form.
\begin{center}
	$env_V,sto \vdash e \rightarrow v$
\end{center} 

\subsection{Big-Step Semantic for Expressions}
An expression is any syntactic component that returns or has a value. In HCL most expressions behave the same, but a few transition-systems uses special notation.
When this is the case, it will be made apparent before that transition-system.

\textbf{\large{General Rules for Expressions}} \\
All expression in the abstract syntax have two rewriting rules in common.
The variable rule and the parenthesis rule.
To avoid the possibility of cluttering the section, these rules are described below, using the metavariable $e \in Exp$, where the syntax category $Exp$ is the set of all expressions.

\textbf{[$Parent_{BSS}$]}
\begin{center}
	\begin{math}
	\cfrac
		{env_V,\ sto \vdash e \rightarrow_e v}
		{env_V,\ sto \vdash (e) \rightarrow_e v}
	\end{math}
\end{center}

\textbf{[$Var_{BSS}$]}
\begin{center}
	\begin{math}
	env_V,\ sto \vdash x \rightarrow_e v
	\texttt{ if } env_V\ x = l
	\texttt{ and } sto\ l = v
	\end{math}
\end{center}

\textbf{\large{Arithmetic Expressions}}\\
Arithmetic expressions returns a value $v$, where $v \in \mathbb{Z}$ or $v \in \mathbb{R}$.
The transitions-system for $A_{exp}$ is $(A_{exp} \cup D, \rightarrow_a, D)$, 
where $D = \mathbb{Z} \cup \mathbb{R}$.
The transitions can be seen below.

\textbf{[$Add_{BSS}$]}\\
\begin{center}
	\begin{math}
	\cfrac
		{env_V,\ sto \vdash a_1 \rightarrow_a v_1 \quad env_V, sto \vdash a_2 \rightarrow_a v_2}
		{env_V,\ sto \vdash a_1 + a_2 \rightarrow_a v}
	\end{math}
	
	
	\texttt{where} $v = v_1 + v_2$
\end{center}

\textbf{[$Minus_{BSS}$]}\\
\begin{center}
	\begin{math}
	\cfrac
		{env_V,\ sto \vdash a_1 \rightarrow_a v_1 \quad env_V, sto \vdash a_2 \rightarrow_a v_2}
		{env_V,\ sto \vdash a_1 - a_2 \rightarrow_a v}
	\end{math}
	
	
	\texttt{where} $v = v_1 - v_2$
\end{center}

\textbf{[$Mult_{BSS}$]}\\
\begin{center}
	\begin{math}
	\cfrac
	{env_V,\ sto \vdash a_1 \rightarrow_a v_1 \quad env_V, sto \vdash a_2 \rightarrow_a v_2}
	{env_V,\ sto \vdash a_1 * a_2 \rightarrow_a v}
	\end{math}
	
	
	\texttt{where} $v = v_1 \cdot v_2$
\end{center}

\textbf{[$Div_{BSS}$]}\\
\begin{center}
	\begin{math}
	\cfrac
	{env_V,\ sto \vdash a_1 \rightarrow_a v_1 \quad env_V, sto \vdash a_2 \rightarrow_a v_2}
	{env_V,\ sto \vdash a_1 / a_2 \rightarrow_a v}
	\end{math}
	
	
	\texttt{where} $v = v_1 / v_2$
\end{center}

\textbf{[$Num_{BSS}$]}\\
\begin{center}
	\begin{math}
		env_V,\ sto \vdash n \rightarrow_a v
	\end{math}
	\texttt{ if } $\mathbb{N}[n] = v$
\end{center}

\textbf{\large{Boolean Expressions}}\\
Boolean expressions returns a value $v$, where $v \in \{True, False\}$.
The transition-system for $B_{exp}$ is $(B_{exp} \cup \{True, False,\} \rightarrow_b, \{True, False\})$.
The transition can be seen below.

\textbf{[$EqualTrue_{BSS}$]}\\
\begin{center}
	\begin{math}
		\cfrac
			{env_V,\ sto \vdash a_1 \rightarrow_a v_1 \quad env_V,\ sto \vdash a_2 \rightarrow_a v_2}
			{env_V,\ sto \vdash a_1 == a_2 \rightarrow_b True}
	\end{math}
	
	\texttt{ if } $v_1 = v_2$
\end{center}

\textbf{[$EqualFalse_{BSS}$]}\\
\begin{center}
	\begin{math}
	\cfrac
	{env_V,\ sto \vdash a_1 \rightarrow_a v_1 \quad env_V,\ sto \vdash a_2 \rightarrow_a v_2}
	{env_V,\ sto \vdash a_1 == a_2 \rightarrow_b False}
	\end{math}
	
	\texttt{ if } $v_1 \neq v_2$
\end{center}

\textbf{[$LessThanTrue_{BSS}$]}\\
\begin{center}
	\begin{math}
	\cfrac
	{env_V,\ sto \vdash a_1 \rightarrow_a v_1 \quad env_V,\ sto \vdash a_2 \rightarrow_a v_2}
	{env_V,\ sto \vdash a_1 < a_2 \rightarrow_b True}
	\end{math}
	
	\texttt{ if } $v_1 < v_2$
\end{center}

\textbf{[$LessThanFalse_{BSS}$]}\\
\begin{center}
	\begin{math}
	\cfrac
	{env_V,\ sto \vdash a_1 \rightarrow_a v_1 \quad env_V,\ sto \vdash a_2 \rightarrow_a v_2}
	{env_V,\ sto \vdash a_1 < a_2 \rightarrow_b False}
	\end{math}
	
	\texttt{ if } $v_1 \not< v_2$
\end{center}

\textbf{[$GreaterThanTrue_{BSS}$]}\\
\begin{center}
	\begin{math}
	\cfrac
	{env_V,\ sto \vdash a_1 \rightarrow_a v_1 \quad env_V,\ sto \vdash a_2 \rightarrow_a v_2}
	{env_V,\ sto \vdash a_1 > a_2 \rightarrow_b True}
	\end{math}
	
	\texttt{ if } $v_1 > v_2$
\end{center}

\textbf{[$GreaterThanFalse_{BSS}$]}\\
\begin{center}
	\begin{math}
	\cfrac
	{env_V,\ sto \vdash a_1 \rightarrow_a v_1 \quad env_V,\ sto \vdash a_2 \rightarrow_a v_2}
	{env_V,\ sto \vdash a_1 > a_2 \rightarrow_b False}
	\end{math}
	
	\texttt{ if } $v_1 \not> v_2$
\end{center}

\textbf{[$LessThanOrEqualTrue_{BSS}$]}\\
\begin{center}
	\begin{math}
	\cfrac
	{env_V,\ sto \vdash a_1 \rightarrow_a v_1 \quad env_V,\ sto \vdash a_2 \rightarrow_a v_2}
	{env_V,\ sto \vdash a_1 <= a_2 \rightarrow_b True}
	\end{math}
	
	\texttt{ if } $v_1 <= v_2$
\end{center}

\textbf{[$LessThanOrEqualFalse_{BSS}$]}\\
\begin{center}
	\begin{math}
	\cfrac
	{env_V,\ sto \vdash a_1 \rightarrow_a v_1 \quad env_V,\ sto \vdash a_2 \rightarrow_a v_2}
	{env_V,\ sto \vdash a_1 <= a_2 \rightarrow_b False}
	\end{math}
	
	\texttt{ if } $v_1 \not<\not= v_2$
\end{center}

\textbf{[$AndTrue_{BSS}$]}\\
\begin{center}
	\begin{math}
	\cfrac
		{env_V,\ sto \vdash b_1 \rightarrow_b True \quad env_V, sto \vdash b_2 \rightarrow_b True}
		{env_V,\ sto \vdash b_1\ and\ b_2 \rightarrow_b True}
	\end{math}
\end{center}

\textbf{[$AndFalse_{BSS}$]}\\
\begin{center}
	\begin{math}
	\cfrac
		{env_V,\ sto \vdash b_1 \rightarrow_b False \quad env_V,\ sto \vdash b_2 \rightarrow_b False}
		{env_V,\ sto \vdash b_1\ and\ b_2 \rightarrow_b False}
	\end{math}
\end{center}

\textbf{[$Not_{BSS}$]}\\
\begin{center}
	\begin{math}
	\cfrac
	{env_V,\ sto \vdash b \rightarrow_b v_2}
	{env_V,\ sto \vdash b\ not \rightarrow_b v_1}
	\end{math}
	
	\texttt{ where } $v_1 \in Bool - v_2$
\end{center}

\textbf{[$TFLit_{BSS}$]}\\
\begin{center}
	\begin{math}
	env_V,\ sto \vdash tf \rightarrow_b v
	\end{math}
	\texttt{ if } $\mathbb{B}[tf] = v$
\end{center}

\textbf{\large{Text Expressions}}\\
Text Expressions returns a value $v$, where $v$ is a sequence of characters from an arbitrary alphabet enclosed in $"$, abbreviated $str$.
In other words, $v$ is a string.
The transition-system for $Txt_{exp}$ is $(Txt_{exp} \cup \{\Sigma^*\},\ \rightarrow_t, \{\Sigma^*\})$.
Here $\Sigma$ is an arbitrary alphabet legal in the context.

\textbf{[$TextConc_{BSS}$]}\\
\begin{center}
	\begin{math}
	\cfrac
		{env_V,\ sto \vdash t_1 \rightarrow_b v_1 \quad env_V,\ sto \vdash t_2 \rightarrow_b v_2}
		{env_V,\ sto \vdash t_1 + t_2 \rightarrow_b v}
	\end{math}
	
	\texttt{ where } $v = v_1\ \circ\ v_2$
\end{center}

\texttt{[$TextToString_{BSS}$]}\\
\begin{center}
	\begin{math}
		\cfrac
			{env_V, sto \vdash e \rightarrow_{e} v'}
			{env_V, sto \vdash e\ toString \rightarrow_{t} v}
	\end{math}
	
	\texttt{ where } $v = "v'"$
\end{center}

\textbf{[$TextLit_{BSS}$]}\\
\begin{center}
	\begin{math}
	env_V,\ sto \vdash t \rightarrow_b v
	\end{math}
	\texttt{ if } $\mathbb{T}[t] = "v"$
\end{center}

\textbf{\Large{List Expressions}}\\
List expressions returns a value $v$, where $v$ consists of a sequence of expression values separated with commas, encapsulated in $[]$.
$v$ is denoted $lst$, and is on the form $[(el \cup Lambda)~]$.
The set of which it is an element of is denoted $Lit$.
The transition-system for $list_{exp}$ is $(list_{exp} \cup List, \rightarrow_{list}, List \cup e)$.
The transitions are shown below.
The length transition returns a value $v$, where $v \in Z$.
Even though this is an exception to the general description of a list expression, the transition has the same format as the rest of the transitions.

\texttt{[$ListLength_{BSS}$]}\\
\begin{center}
	\begin{math}
		\cfrac
			{env_V, sto \vdash num \rightarrow_{a} v}
			{env_V, sto \vdash list\ length \rightarrow_{list} v}
	\end{math}
	
	\texttt{ where } $num = |list|$
\end{center}

\texttt{[$ListAt_{BSS}$]}\\
\begin{center}
	\begin{math}
		\cfrac
			{env_V, sto \vdash a \rightarrow_{a} v' \quad env_V, sto \vdash ls_v' \rightarrow_e v}
			{env_V, sto \vdash list\ @\ a \rightarrow_{list} v}
	\end{math}
	
	\texttt{ where } $ls_v'\ is\ the\ v'th\ element\ of\ list$
\end{center}

\textbf{[$ListLit_{BSS}$]}\\
\begin{center}
	\begin{math}
	env_V,\ sto \vdash lst \rightarrow_{list} v
	\end{math}
	\texttt{ if } $\mathbb{LST}[lst] = v$
\end{center}

\textbf{\Large{Tupple Expressions}}\\
Tupple expressions returns a value $v$, where $v$ consists of a sequence of expression values separated with commas, encapsulated in $()$.
$v$ is denoted $tpl$, and is on the form $((el \cup Lambda)~)$.
The set of which it is an element of is denoted $Tupple$.
The transition-system for $tupple_{exp}$ is $(tupple_{tupple_exp} \cup Tupple, \rightarrow_{tupple}, Tupple)$.
The transitions are shown below.

\texttt{[$TplAt_{BSS}$]}\\
\begin{center}
	\begin{math}
	\cfrac
	{env_V, sto \vdash a \rightarrow_{a} v' \quad env_V, sto \vdash tp_v' \rightarrow_e v}
	{env_V, sto \vdash tupple\ @\ a \rightarrow_{tupple} v}
	\end{math}
	
	\texttt{ where } $tp_v'\ is\ the\ v'th\ element\ of\ tupple$
\end{center}

\textbf{[$ListLit_{BSS}$]}\\
\begin{center}
	\begin{math}
	env_V,\ sto \vdash tpl \rightarrow_{tupple} v
	\end{math}
	\texttt{ if } $\mathbb{TPL}[tpl] = v$
\end{center}

\subsection{Big-Step Semantic for Statements}
Statements $Stm$ can change the change the store, because they can change the value of an existing variable.
The transition-system for $stm$ is $(Stm\ X\ Sto\ X \cup Sto, \rightarrow_{stm}, Sto)$.
Given the variable bindings $env_V$ and the procedure bindings $env_P$, the execution of the statement $S$ will 
result in $sto$ changing to $sto'$.

This means that the transitions must be on the form.

\begin{center}
	\begin{math}
		{env_V, env_P \vdash \langle S, sto \rangle \rightarrow_{stm} sto'}
	\end{math}
\end{center}

The transitions are shown below.

\texttt{[$Ass_{BSS}$]}\\
\begin{center}
	\begin{math}
		{env_V, env_P \vdash \langle x = e, sto \rangle \rightarrow_{stm} sto[l \vdash v]}
	\end{math}
	
	\texttt{ where } $env_V, sto \vdash e \rightarrow_e v$
	\texttt{ and } $env_V\ x = l$
\end{center}

\texttt{[$Comp_{BSS}$]}\\
\begin{center}
	\begin{math}
		\cfrac
			{env_V, env_P \vdash \langle S_1, sto \rangle \rightarrow_{stm} sto'' \quad env_V, env_P \vdash \langle S_2, sto'' \rangle \rightarrow_{stm} sto'}
			{env_V, env_P \vdash \langle S_1; S_2, sto \rangle \rightarrow_{sto'}}
	\end{math}
\end{center}

\texttt{[$Then-True_{BSS}$]}\\
\begin{center}
	\begin{math}
		\cfrac
			{env_V, env_P \vdash \langle L,sto \rangle \rightarrow_{stm} sto'}
			{env_V, env_P \vdash \langle b\ then\ L, sto \rangle \rightarrow_{stm} sto'}
	\end{math}
	
	\texttt{ if } $env_V, sto \vdash b \rightarrow_b True$
\end{center}

\texttt{[$Then-False_{BSS}$]}\\
\begin{center}
	\begin{math}
	\cfrac
	{env_V, env_P \vdash \langle \epsilon,sto \rangle \rightarrow_{stm} sto}
	{env_V, env_P \vdash \langle b\ then\ L, sto \rangle \rightarrow_{stm} sto}
	\end{math}
	
	\texttt{ if } $env_V, sto \vdash b \rightarrow_b False$
\end{center}

\texttt{[$While-False_{BSS}$]}\\
\begin{center}
	\begin{math}
		{env_V, env_P \vdash \langle L\ while\ b, sto \rangle \rightarrow_{stm} sto}
	\end{math}
	
	\texttt{ if } $env_V, sto \vdash b \rightarrow_b False$
\end{center}

\texttt{[$While-True_{BSS}$]}\\
\begin{center}
	\begin{math}
		\cfrac
			{env_V, env_P \vdash \langle L, sto \rangle \rightarrow_{stm} sto'' \quad env_V, env_P \vdash \langle L\ while\ b, sto'' \rangle \rightarrow_{stm} sto'}
			{env, env_P \vdash \langle L\ while\ b, sto \rangle \rightarrow_{stm} sto'}
	\end{math}
\end{center}

\texttt{[$ProcCall-NoParams_{BSS}$]}\\
\begin{center}
	\begin{math}
		\cfrac
			{env_v'[x \vdash l][nest \vdash new\ l], env_P'[p \vdash (L, e, env_V', env_P')] \vdash \langle L, sto[l \vdash v] \rangle \rightarrow_{stm} sto'}
			{env_v, env_P \vdash \langle p, sto \rangle \rightarrow_{stm} sto'}
	\end{math}
	
	\texttt{ where } $env_P\ p = (L, e, env_V', env_P')$\\
	\texttt{ and } $ env_V', sto \vdash e \rightarrow_{e} v$\\
	\texttt{ and } $l = env_V\ next$
\end{center}

\texttt{[$ProcCall-Params_{BSS}$]}\\
\begin{center}
	\begin{math}
	\cfrac
	{env_v'[x \vdash l][nest \vdash new\ l], env_P'[p \vdash (L, x\sim, e, env_V', env_P')] \vdash \langle L, sto[l \vdash v] \rangle \rightarrow_{stm} sto'}
	{env_v, env_P \vdash \langle x\ p\ x\sim, sto \rangle \rightarrow_{stm} sto'}
	\end{math}
	
	\texttt{ where } $env_P\ p = (L, x\sim, e, env_V', env_P')$\\
	\texttt{ and } $ env_V', sto \vdash e \rightarrow_{e} v$\\
	\texttt{ and } $l = env_V\ next$
\end{center}

\texttt{[$VarDclStm_{BSS}$]}\\
\begin{center}
	\begin{math}
		\cfrac
			{\langle D_V, env_V, sto \rangle \rightarrow_{D_V} (env_V', sto')}
			{env_V, env_P \vdash \langle D_V, sto \rangle \rightarrow_{stm} sto'}
	\end{math}
\end{center}

\texttt{[$ProcDclStm_{BSS}$]}\\
\begin{center}
	\begin{math}
		\cfrac
			{env_V' \vdash \langle D_P, env_P \rangle \rightarrow_{D_P} env_P'}
			{env_V, env_P \vdash \langle D_P, sto \rangle \rightarrow_{stm} sto'}
	\end{math}
\end{center}

\texttt{[$EmptyStm_{BSS}$]}\\
\begin{center}
	\begin{math}
		{env_V, env_P \vdash \langle \epsilon, sto \rangle \rightarrow_{stm} sto}
	\end{math}
\end{center}

\subsection{Big-Step Semantic for Variable Declarations}
Variable declarations $Dcl_V$ can change the variable environment $Env_v$ by declaring new variables in $Var$.
They can also change the store by binding variables to new addresses.

The transition-system for $Dcl_V$ is ($(Dcl_V\ X\ Env_V\ X\ Sto) \cup Env_V\ X\ Sto, \rightarrow_{D_{V}}, Env_V\ X\ Sto$).

This means that the transitions must be on the form.

\begin{center}
	$\langle D_V,env_V,sto \rangle \rightarrow_{D_V} (env_V^{'} , sto^{'})$
\end{center}

The transitions are shown below.

\textbf{[$Dcl_VAssExp_{BSS}$]}\\
\begin{center}
	\begin{math}
	\cfrac
		{\langle D_V,\ env_V[x \mapsto l][next \mapsto l'],\ sto[l \mapsto v] \rangle \rightarrow_{D_V} (env_V',\ sto')}
		{\langle T\ x = e;\ S,\ env_V,\ sto \rangle \rightarrow_{D_V} (env_V',\ sto')}
	\end{math}
	
	\texttt{ where } $env_V,\ sto \vdash e \rightarrow_e v$
	\texttt{ and } $l = env_V\ next$
	\texttt{ and } $l' = new\ l$
\end{center}

\texttt{[$Dcl_VAssFnc_{BSS}$]}\\
\begin{center}
	\begin{math}
		\cfrac
			{\langle D_V,\ env_V[x \mapsto l][next \mapsto l'],\ sto[l \mapsto v] \rangle \rightarrow_{D_V} (env_V',\ sto')}
			{\langle T\ x = f;\ S,\ env_V,\ sto \rangle \rightarrow_{D_V} (env_V',\ sto')}		
	\end{math}
	
	\texttt{ where } $env_V,\ sto \vdash f \rightarrow_{call_{exp}} v$
	\texttt{ and } $l = env_V\ next$
	\texttt{ and } $l' = new\ l$
\end{center}

\textbf{[$Dcl_VNoAss_{BSS}$]}\\
\begin{center}
	\begin{math}
	\cfrac
		{\langle D_v,\ env_v[x \mapsto l][next \mapsto l'],\ sto \rangle \rightarrow_{D_V} (env_v',\ sto)}
		{\langle T\ x;\ S,\ env_v,\ sto \rangle \rightarrow_{D_V} (env_v',\ sto)}
	\end{math}
	
	\texttt{ where } $l = env_v\ next$
	\texttt{ and } $l' = new\ l$
\end{center}

\textbf{[$Dcl_VEmpty_{BSS}$]}\\
\begin{center}
	\begin{math}
	\langle \epsilon,\ env_v,\ sto \rangle \rightarrow (env_v,\ sto)
	\end{math}
\end{center}

\subsection{Big-Step Semantics for Procedure and Function Declarations}
Function declarations $Dcl_P$ can change the procedure environment $Env_P$ by declaring new functions in $Fnames$.

The transition-system for $Dcl_P$ is ($Fnames \rightharpoonup Stm\ X\ Var\ X\ Exp\ X\ Env_V\ X\ Env_P$).
This means that the transitions must be on the form.

\begin{center}
	$\langle D_P, env_V, env_P \rangle \rightarrow_{D_P} (env_P)$
\end{center}

Procedures are functions with no return value and therefore are statements.
Any function returning a value is an expression.
For simplicity, all functions will be denoted f in the function declaration transitions.

Declarations of functions must remember the expression directly following the return keyword, this is accomplished by binding the function in the procedure environment with this expression.
This is not the case for procedures, since they do not return a value.
The rules for procedures and functions are the same, except for the value $e$ being an expression in functions, it is $none$ in procedure rules.

The transitions are shown below.

\texttt{[$Dcl_PAss_{BSS}$]}\\
\begin{center}
	\begin{math}
		\cfrac
			{env_V \vdash \langle S, env_P[f \vdash (L, x\sim, e, env_V, env_P)] \rangle \rightarrow_{D_P} env_P'}
			{env_v \vdash \langle T\ f\ =\ ([T\ x]\sim):\ T\ L;\ S,env_P \rangle \rightarrow_{D_P} env_P'}
	\end{math}
\end{center}

\texttt{[$Dcl_PNoAss_{BSS}$]}\\
\begin{center}
	\begin{math}
	\cfrac
	{env_V \vdash \langle S, env_P[f \vdash (L, x\sim, e, env_V, env_P)] \rangle \rightarrow_{D_P} env_P'}
	{env_v \vdash \langle T\ f;\ S,env_P \rangle \rightarrow_{D_P} env_P'}
	\end{math}
\end{center}

\texttt{[$Dcl_Pempty_{BSS}$]}\\
\begin{center}
	\begin{math}
		{env_V \vdash \langle \epsilon, env_P \rangle \rightarrow_{D_P}\ env_P'}
	\end{math}
\end{center}

In conclusion, HCL is best described as a functional language with side effects. Because of this it is possible to define the semantics using operational structural semantics.
HCL follows a structural like pattern in most of it's functionality, excluding the high-order functionality present in the syntax.
The semantics presented in this section should give any reader familiar with operational semantics the proper knowledge to fully understand the functionality of HCL.
Now that the semantics of HCL have been defined, it is now relevant to define the type rules of HCL.