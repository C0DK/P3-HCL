% !TeX root = ../main.tex

\subsection{Traversing the AST}
The code generation phase of the compiler traverses the AST and generates code according to the nodes in the AST.
Usually, traversal of the tree is done in a depth first manner, and code is generated based on visited nodes.
Traditionally, the AST is traversed by utilizing the visitor pattern. 
However, more modern languages, especially functional oriented languages, have introduced the concept of pattern matching.
Pattern matching allows for a short and concise way of traversing the tree.

When possible, the pattern matching method is often preferred over the traditional visitor pattern. 

Consider an AST for the following code:
$$x = 5 + 7$$
As well as a simple traversal using visitor pattern, shown in listing \ref{lis:visitorPattern}, and pattern matching, shown in listing \ref{lis:patternMatching}.

\begin{lstlisting}[language=Kotlin,label=lis:visitorPattern,caption=Tree traversal using the visitor pattern.]
interface IAstNodes {
	fun accept(visitor: IVisitor)
}

sealed class AstNodes {
	data class Identifier(val id: String): AstNodes(), IAstNodes {
		override fun accept(visitor: IVisitor) {
			visitor.visit(this)
		}
	}
	sealed class Expression: IAstNodes {
		data class Number(val number: Int): Expression(), IAstNodes {
			override fun accept(visitor: IVisitor) {
				visitor.visit(this)
			}
		}
		data class Addition(val left: Expression, val right: Expression) : Expression(), IAstNodes {
			override fun accept(visitor: IVisitor) {
				visitor.visit(this)
			}
		}
	}
	data class Assignment(val identifier: Identifier, val expression: Expression): AstNodes(), IAstNodes {
		override fun accept(visitor: IVisitor) {
			visitor.visit(this)
		}
	}
}

interface IVisitor {
	fun visit(identifier: AstNodes.Identifier)
	fun visit(number: AstNodes.Expression.Number)
	fun visit(addition: AstNodes.Expression.Addition)
	fun visit(assignment: AstNodes.Assignment)
}

class PrintVisitor(private val sb: StringBuilder): IVisitor {
	override fun visit(identifier: AstNodes.Identifier) {
		sb.append(identifier.id)
	}

	override fun visit(number: AstNodes.Expression.Number) {
		sb.append("${number.number}")
	}

	override fun visit(addition: AstNodes.Expression.Addition) {
		addition.left.accept(this)
		sb.append(" + ")
		addition.right.accept(this)
	}

	override fun visit(assignment: AstNodes.Assignment) {
		assignment.identifier.accept(this)
		sb.append(" = ")
		assignment.expression.accept(this)
	}
}

fun main(args: Array<String>) {
	val node =
		Assignment(
			Identifier("x"),
			Expression.Addition(
				Expression.Number(5),
				Expression.Number(7)
			)
		)
	val output = StringBuilder()
	val visitor = PrintVisitor(output)
	node.accept(visitor)
	println(output.toString())
}
\end{lstlisting}

\begin{lstlisting}[language=Kotlin,label=lis:patternMatching,caption=Tree traversal using pattern matching.]
sealed class AstNodes {
	data class Identifier(val id: String) : AstNodes()
	sealed class Expression : AstNodes() {
		data class Number(val number: Int) : Expression()
		data class Addition(val left: Expression, val right: Expression) : Expression()
	}
	data class Assignment(val identifier: Identifier, val expression: Expression) : AstNodes()
}

fun AstNodes.format(): String = when (this) {
	is AstNodes.Assignment -> "${identifier.format()} = ${expression.format()}"
	is AstNodes.Identifier -> id
	is AstNodes.Expression.Number -> "${number}"
	is AstNodes.Expression.Addition -> "${left.format()} + ${right.format()}"
}

fun main(args: Array<String>) {
	val node =
		AstNodes.Assignment(
			AstNodes.Identifier("x"),
			AstNodes.Expression.Addition(
				AstNodes.Expression.Number(5),
				AstNodes.Expression.Number(7)
			)
		)
	println(node.format())
}
\end{lstlisting}

The pattern matching version is significantly shorter, and the difference in size only increases, as more AST nodes need to be handled.

Since Kotlin allows for pattern matching, it is deemed the ideal way to implement the traversal of the AST in the HCL compiler.

Though both the code to be generated and the AST nodes are different in the actual HCL compiler, the general tree traversal method is identical to the method shown in \ref{lis:patternMatching}.

