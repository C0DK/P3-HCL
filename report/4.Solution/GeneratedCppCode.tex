% !TeX root = ../main.tex

\subsection{Generated C++ code}
\label{sec:gencplusplus}
While the previous section is primarily focused on the code generation from an abstract point of view, this section focuses on the generated \texttt{C++} code.
Since \texttt{C++} is the output language of the HCL compiler, all the language concepts of HCL have to be represented in valid \texttt{C++} code.

\textbf{Primitive types}
The primitive types of HCL includes numbers, text and booleans. 
Booleans are straightforward, as HCL's boolean behaves exactly like the \texttt{C++} boolean type.
This means that the regular \texttt{C++} boolean type can be used as the output type of HCL booleans.

As HCL aims for simplicity, only one number type is introduced. 
This type needs to be able to store both integers and non-integers. 
The simple solution is to convert all HCL numbers into \texttt{C++}'s double type.
The solution with doubles is not ideal, but it will fulfill the requirements of the number type in HCL.

\texttt{C++} does have an implementation of a string type in its standard library.
However, the Arduino does not include this implementation, but rather uses its own.

As the string type shares a lot of functionality with that of lists, the representation of strings in the generated \texttt{C++} code, is based on the ConstList.

\textbf{Lists}

For lists, \texttt{C++} includes a list type in its standard library called vector. 
This vector type is, like the string type, not available in the libraries provided by the Arduino.
Furthermore, since the vector implementation is mutable and the lists in HCL are immutable, a simple implementation of a constant list is implemented instead.

One thing to be aware of when using lists, is that the size of the list cannot always be determined at compile time.
This means that lists will usually do heap allocation. 
With heap allocation comes the need to do proper memory handling, eg.
clean up allocated memory in order to avoid memory leaks.

Memory handling is done by utilizing the \texttt{C++} concept of RAII\footnote{Resource acquisition is initialization} along with \texttt{C++} smart pointers.
This means that a pointer is implemented, which keeps a count of the references to the memory pointed to.
If the counter goes to zero, this means that there are no more references to memory, and the memory may be deallocated safely.

It should be noted, that the current HCL compiler uses the std::shared\_ptr, which is not available in the Arduino standard library.
The implementation should however be easily ported to the Arduino when needed by implementing a similar smart pointer ourself.

\textbf{Tuples}

Tuples are represented as type defined structures in \texttt{C++}.
HCL has a theoretically infinite amount of tuples, so obviously not all tuples can be generated.
As mentioned in the previous section, the type generator will traverse the AST in order to find all the tuples used within the program to be compiled.
For each defined tuple a few helper methods are also generated.
This is the \textit{\textbf{toText}} function, along with accessing methods for each element in the tuple.

\textbf{Functions}

Functions are one of the most interesting and advanced parts of the code generation.
This is because closures, per their nature, are advanced concepts.
They have to store information about the content of their body along with scope information.
They also have to potentially be able to expand the life time of scoped variables, if the variables are referenced from within the body of the closure.
Fortunately, the \texttt{C++} standard library includes a std::function data type, which can store the closures that is part of the \texttt{C++} language.
However, once again, this is not a part of the Arduino standard library.

Because of time constraints, the current HCL compiler is only implemented using the std::function data type, and not a custom lambda type that is compatible with the Arduino.
This means that the generated code is not actually compatible with the Arduino at this point in time.

There is one limitation to the std::function combined with the closures in \texttt{C++}.
This is the problem known as dangling references.
The problem occurs when a closure is returned, and executed after a referenced variable has gone out of scope.
The reference now points to some unallocated garbage memory, which leaves undefined behavior in the program.

Although this feature would be nice to have in HCL, it also has to be deemed as undefined behavior for HCL when the target language is \texttt{C++} and the closures are implemented using the method mentioned above.

\textbf{Generics}

As for generic types, which is used in functions and lists, \texttt{C++} has a language concept known as templates.
Templates are \texttt{C++}'s version of generics, and works by determining all the template variants used and then creating functions according to these variants.
Templates are not as powerful as modern generics, for example C\#'s generics, but they are enough for HCL, as HCL's generics are also not that powerful.
One lacking feature of both, is the ability to create an instance of type T.

\textbf{Naming of outputted types and variables}

Aside from storing the types appropriately, there are a few other issues when generating code for \texttt{C++}
An interesting thing to mention is the translation of names from HCL to \texttt{C++}.
In HCL a lot of special characters are allowed for identifiers. 
This includes "+" and "-", both of which are reserved keywords in \texttt{C++}.
The chosen solution for this problem is hashing of all the names HCL and prefixing the hash with, for instance with "IDT" for identifiers or "TPL" for tuples.

This solution leaves HCL vulnerable to hash code collisions.
HCL uses a 32bit hash code, along with the prefixes.
Therefore hashing is a valid solution for now.
However, there may exist more sophisticated ways of solving this issues, that could be a potential improvement for future versions of the HCL compiler. 
