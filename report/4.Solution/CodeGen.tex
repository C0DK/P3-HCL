% !TeX root = ../main.tex
\section{Code Generation}
This section describes the methodology and implementation of the code generation aspect of the compiler for HCL. 

During generation the HCL code is translated into \texttt{C++}, which is subsequently compiled using a \texttt{C++} compiler.
The \texttt{C++} code contains preprocessor directives to make sure it works cross-platform on Arduino, Windows, and Linux.

\subsection{Built-in Functions}
HCL includes several built-in functions that map the languages functionality directly to its \texttt{C++} equivalent. 
These functions are considered to be the most basic functionality of the language, and their use is often required in order to specify more complex functions. 
This section describes the built-in functions of HCL, their implementation and their functionality.

\textbf{Build Functions}\\
The implementation of all built-in functions is handled by the \texttt{buildFunction} function seen in snippet \ref{lis:buildFunction}.

\begin{lstlisting}[language=Kotlin,label=lis:buildFunction,caption=The implementation of buildFunction.]
private fun buildFunction(identifier: String, parameters: List<Parameter>, returnType: Type, body: String, attributes: LambdaExpressionAttributes = BuiltinLambdaAttributes) =
    AstNode.Command.Declaration(returnType, identifier.asIdentifier(),
	    AstNode.Command.Expression.LambdaExpression(
            paramDeclarations = parameters.map {
                AstNode.ParameterDeclaration(it.type, it.identifier.asIdentifier())
            },
            returnType = returnType,
            attributes = attributes,
            body = body.asRawCppLambdaBody()
        )
    )
\end{lstlisting}
The function returns a function-declaration with specified identifier, return-type and lambda-body, where the lambda-body is composed of raw \texttt{C++} code. 
Using the implementation to build functions is then simply a manner of passing in the required arguments, as seen in snippet \ref{lis:thenFunction}.

\begin{lstlisting}[language=Kotlin,label=lis:thenFunction,caption=built-in 'then' function implemented using \texttt{buildFunction}.]
private fun buildThenFunction() = buildFunction(
        identifier = "then",
        parameters = listOf(
                Parameter("condition", Type.Bool),
                Parameter("body", Type.Func.ExplicitFunc(listOf(Type.None))
        ),
        returnType = Type.Bool,
        body = "if (condition) { body(); }\nreturn condition;"
)
\end{lstlisting}
The above snippet shows how the built-in function \texttt{then} is declared using the \texttt{buildFunction} implementation. 
The arguments provided specify the identifier for the function (2), the parameters for the function (3), namely a condition of type \texttt{bool} and a body of type \texttt{ExplicitFunc} without a return-type.
It also specifies that the return-type of the function itself is a \texttt{bool}(7) and then presents the body of the function in raw \texttt{C++} code (8). 

In order to follow the DRY\footnote{Don't repeat yourself (DRY) is a principle aimed at reducing code repetition in software development.
} principles of coding, a helper-function for implementing simple arithmetic functions is provided in snippet \ref{lis:buildOperator}.
\begin{lstlisting}[language=Kotlin,label=lis:buildOperator,caption=The implementation of buildOperator.]
private inline fun<reified V, reified H, reified R>buildOperator(functionName: String, operator: String = functionName)
        where V : Type, H : Type, R : Type = buildFunction(
        identifier = functionName,
        parameters = listOf(
                Parameter("leftHand", V::class.objectInstance!!),
                Parameter("rightHand", H::class.objectInstance!!)
        ),
        returnType = R::class.objectInstance!!,
        body = "return leftHand $operator rightHand;"
)
\end{lstlisting}
The \texttt{buildOperator} function is a generic function that is called with specified types for input-parameters and return-type. 
The function uses those types to specify the arguments for the \texttt{buildFunction} function.

All built-in functions are listed in appendix at \ref{builtinAppend}.

\subsection{Generator Classes}

The overall code generation is done within the \texttt{ProgramGenerator} class.
The class' responsibility is limited to calling the three predominant generators, namely \texttt{CodeGenerator}, \texttt{TypeGenerator} and \texttt{MainGenerator}.
In addition to calling the other classes, the \texttt{ProgramGenerator} class also adds return code to the AST to make sure that the program returns when terminated. 

Snippet \ref{lis:programGen} shows the \textit{generate} function from \texttt{ProgramGenerator}.

\begin{lstlisting}[language=Kotlin,label=lis:programGen,caption=The implementation of \textit{generate} in \texttt{ProgramGenerator}.]
override fun generate(ast: AbstractSyntaxTree): List<FilePair> = 
	listOf(
		HelperHeaders.constList,
		HelperHeaders.ftoa,
		FilePair("builtin.h", CodeGenerator().generate(ast.builtins())),
		FilePair("types.h", TypeGenerator().generate(ast)),
		FilePair("main.cpp", MainGenerator().generate(ast.addReturnCode().notBuiltins()))
)
\end{lstlisting}

The function first creates header files for the helper classes \textit{constList}(3) and \textit{ftoa}(4), which allow the use of the list-type in \texttt{C++}, and for converting doubles to text. 
The function then creates the header file for the built-in functions of the language(5).
When the \textit{generate} function within the \texttt{CodeGenerator} class is called with the built-in functions from the language, it formats the functions to resemble valid \texttt{C++} function declarations, whilst also adding descriptive comments of its functionality.
The resulting text is written to a file that serves as a header-file for the final program. \\
The next line of code (6) calls the \textit{generate} function from the \texttt{TypeGenerator} class.
This class is responsible for formatting tuples in the language by creating equivalent \textit{struct} declarations in \texttt{C++}.
The functions fetch all tuples within the inputted AST, formats them, and writes the result of the formatting to a header file.
Before returning, the function also generates functions for each tuple found.
These generated functions allow the tuples to be printed using the \textit{toText} function, and also allow the tuples to return a specific element from within themselves.\\
The last function that is called is the \texttt{MainGenerator} class' \textit{generate} function (7). 
This last \textit{generate} function is responsible for wrapping and setting up Arduino's \texttt{main}, \texttt{setup} and \texttt{loop} functions.
Before generation, a return code is added to the AST, to ensure that the program has a final return.
Built-in functions are not added to generation, as these have already been handled. 

The \texttt{MainGenerator}'s \textit{generate} function is described in detail in snippet \ref{lis:mainGen}.
\begin{lstlisting}[language=Kotlin,label=lis:mainGen,caption=The implementation of \textit{generate} in \texttt{MainGenerator}.]
override fun generate(ast: AbstractSyntaxTree): String {
	val stringBuilder = StringBuilder(mainHeader)
	val declarations = ast.genFromFilterWithMap ({ it.isDecl }, 
	{
		// Find all declarations	
	})
	val setup = ast.genFromFilterWithMap ({ !it.isLoop && !it.isDecl }, 
	{
	    // Assign all declarations
	})
	val loop = ast.genForLoop()
	stringBuilder.appendln(declarations)
	stringBuilder.appendln(setup.wrapSetup())
	stringBuilder.appendln(loop.wrapLoop())

	stringBuilder.appendln(("setup();\n" +
	(if (loop.isNotBlank()) "while(1) { loop(); }\n" else "")).wrapMain()
	)
		return stringBuilder.toString()
	}
\end{lstlisting}
(2) Add all header files and namespaces.\\ 
(3) Find all declarations in source code.\\
(7) Find all commands in source code that are not declarations or associated with the \textbf{loop} function.\\
(11) Find all commands that are associated with the \textbf{loop} function.\\
(12) Add all declarations to output file.\\
(13) Wrap and add \textbf{setup} function to file.\\
(14) Wrap and add \textbf{loop} function to file.\\
(16) Wrap \textbf{main} function and add call to \textbf{setup} function.\\
if \textbf{loop} function is not empty, a call to it is also added to the \textbf{main} function.\\
(19) Source code is generated and returned from function. 

Once all files has been created the code generation is completed.
The resulting files are translated in full into \texttt{C++}, and can be used together with a native \texttt{C++} compiler or run through the Arduino editor.

