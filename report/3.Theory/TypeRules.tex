% !TeX root = ../main.tex
%Content includes all type rules in HCL
%Should be a relatively short section, since much of the content is self explanatory.
\section{Type Rules}
To ensure that any program written in HCL runs correctly after compilation, the language follows certain type rules.


\subsection{Type Rules for Expressions}
%quick introduction.
\textbf{Addition}\\
\begin{center}
	\begin{math}
	\cfrac
	{E \vdash e_1: T \quad E \vdash e_2: T}
	{E \vdash e_1 + e_2 : T}
	\end{math}\\[1\baselineskip]
	\texttt{where} $T \in \{num, txt, list\}$
\end{center}

\textbf{Subtraction}\\
\begin{center}
	\begin{math}
	\cfrac
	{E \vdash e_1: num \quad E \vdash e_2: num}
	{E \vdash e_1 - e_2 : num}
	\end{math}
\end{center}

\textbf{Multiplication}\\
\begin{center}
	\begin{math}
	\cfrac
	{E \vdash e_1: num \quad E \vdash e_2: num}
	{E \vdash e_1 * e_2 : num}
	\end{math}
\end{center}

\textbf{Division}\\
\begin{center}
	\begin{math}
	\cfrac
	{E \vdash e_1: num \quad E \vdash e_2: num}
	{E \vdash e_1 / e_2 : num}
	\end{math}
\end{center}

\textbf{Less than}\\
\begin{center}
	\begin{math}
	\cfrac
	{E \vdash e_1: num \quad E \vdash e_2: num}
	{E \vdash e_1 < e_2 : bool}
	\end{math}
\end{center}

\textbf{Greater than}\\
\begin{center}
	\begin{math}
	\cfrac
	{E \vdash e_1: num \quad E \vdash e_2: num}
	{E \vdash e_1 > e_2 : bool}
	\end{math}
\end{center}

\textbf{Equals}\\
\begin{center}
	\begin{math}
	\cfrac
	{E \vdash e_1: T \quad E \vdash e_2: T}
	{E \vdash e_1\ equals\ e_2 : bool}
	\end{math}\\[1\baselineskip]
	\texttt{where} $T \in \{num, bool, txt, list\}$
\end{center}

\textbf{And}\\
\begin{center}
	\begin{math}
	\cfrac
	{E \vdash e_1: bool \quad E \vdash e_2: bool}
	{E \vdash e_1\ and\ e_2 : bool}
	\end{math}
\end{center}

\textbf{Or}\\
\begin{center}
	\begin{math}
	\cfrac
	{E \vdash e_1: bool \quad E \vdash e_2: bool}
	{E \vdash e_1\ or\ e_2 : bool}
	\end{math}
\end{center}

\textbf{Function call}\\
\begin{center}
	\begin{math}
	\cfrac
	{E \vdash e_1 : T_1 \ .
	.
	.
	\ E \vdash e_k : T_k}
	{E \vdash e_1\ f\ e_2 \ .
	.
	.
	\ e_k : T_f}
	\end{math}
	\\[1\baselineskip]
	\texttt{where} $e_1\ .
	.
	.
	\ e_k$ are the expressions passed to function $f$.
	\\
	\texttt{and} $T_1\ .
	.
	.
	\ T_k$ are the types of the formal parameters $f$.
	\\
	\texttt{and} $T_f$ is the return type of $f$.
	\\
	\texttt{and} $k >= 0$
\end{center}

\textbf{Variable declaration}\\
\begin{center}
	\begin{math}
		\cfrac
		{E \vdash e : T}
		{E \vdash T x = e \rightarrow ok}
	\end{math}
\end{center}

\textbf{Explicit function declaration}\\
\begin{center}
	\begin{math}
		\cfrac
		{E \vdash U_1 : T_1\ 
		.
		.
		.
		\ E \vdash U_k : T_k}
		{E \vdash func[T_1\
			.
			.
			.
			\ T_k]\ f = (U_1\ p_1\
			.
			.
			.
			\ U_{k-1}\ p_{k-1}): U_k\ \{ S \} \rightarrow ok}
	\end{math}
\end{center}


\subsection{Type Rules for Statements}
%quick introduction.
hej
%Conclusion.

