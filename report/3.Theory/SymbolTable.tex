%THIS SECTION IS SUPPOSED TO BE A SUBSECTION TO THE PARSER SECTION.
%Should present the interface, which all symbol table implementations must implement.
%Should clarify all considerations through the design process.
\subsection{Symbol Table}
This section concerns with the implementation of the symbol table. 
first, the interface implemented by the the symbol table will be presented.
Second, all considerations through the design process will be discussed.
This will serve as the reasoning behind the final implementation details.
Last, the problematic aspect of method overloading will be touched upon.

\textbf{Symbol Table Interface}\\
The interface consists of five methods as seen in snippet 3.6 on page \pageref{lis:STInterface}.

The methods on line 2 and 3 concerns with the handling of scopes.
Scopes are handled by a stack data structure.
Intuitively each element on the stack consists of a local symbol table concerning the current scope.

$openScope()$ opens a new scope in the symbol table.
This is done by pushing a new sub-symbol table onto the stack.

$closeScope()$ closes the most recently opened sub-symbol table.
This is done by popping the table from the stack.
Symbol references subsequently revert to outer scopes.

The methods on line 4 and 5 concerns with the handling of entering and retrieval of symbol information.
Since retrieval is the most heavily utilized method in the symbol table implementation, it is essential that time spent on the procedure is minimal.
Because of this all symbol tables uses a hash table data structure for insertion and retrieval of symbols. 

$enterSymbol(name: String, type: TreeNode.Type): Boolean$ Enters a new symbol into the current scope on top of the stack.
The \texttt{name} parameter of type \texttt{String} must be matched with the identifier name in question.
The \texttt{type} parameter of type \texttt{Treenode.Type} must be matched with a valid AST-node type.
The method returns a boolean value indicating whether the operation was successful.

$retrieveSymbol(name: String): Symbol$ Retrieves the symbol stored in the top sub-symbol table on the stack with the name indicated by the \texttt{name} parameter of type \texttt{String}.

The method on line 6 concerns with the testing whether a symbol is declared in the local scope.

$declaredLocally(name: String): Boolean$ Tests whether name is present in the symbol table's current scope.

\begin{lstlisting}[language=java,label=lis:STInterface,caption=The interface which all symbol table implementations must implement.]
interface ISymbolTable{
	fun openScope()
	fun closeScope()
	fun enterSymbol(name: String, type: TreeNode.Type): Boolean
	fun retrieveSymbol(name: String): Symbol
	fun declaredLocally(name: String): Boolean
}
\end{lstlisting}

\textbf{Considerations}\\
%conclusions.