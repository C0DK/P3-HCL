%THIS SECTION IS SUPPOSED TO BE A SUBSECTION TO THE PARSER SECTION.
%Should present the interface, which all symbol table implementations must implement.
%Should clarify all considerations through the design process.
\subsection{Symbol Table}
\label{sec:symbolTable}
This section concerns with the implementation of the symbol table. 
Firstly, the interface implemented by the symbol table will be presented.
Secondly, all considerations through the design process will be discussed.
This will serve as the reasoning behind the final implementation details.

\textbf{Symbol Table Interface}\\
The interface consists of five functions as seen in snippet \ref{lis:STInterface}.

\begin{lstlisting}[language=java,label=lis:STInterface,caption=The interface which all symbol table implementations must implement.]
interface ISymbolTable{
	fun openScope()
	fun closeScope()
	fun enterSymbol(name: String, type: TreeNode.Type): Boolean
	fun retrieveSymbol(name: String): Symbol
	fun declaredLocally(name: String): Boolean
}
\end{lstlisting}

The methods on line 2 and 3 concerns with the handling of scopes.
Scopes are handled by a stack data structure.
Intuitively, each element on the stack consists of a local symbol table concerning the current scope.

The $openScope()$ method opens a new scope in the symbol table.
This is done by pushing a new sub-symbol table onto the stack.

The $closeScope()$ method closes the most recently opened sub-symbol table.
This is done by popping the table from the stack.
Symbol references subsequently revert to outer scopes.

The methods on line 4 and 5 concerns with the handling of entering and retrieval of symbol information.
Since retrieval is the most heavily utilized method in the symbol table implementation, it is essential that time spent on the procedure is minimal.
Because of this, all symbol tables uses a hash table data structure for insertion and retrieval of symbols. 

The $enterSymbol(name: String, type: TreeNode.Type): Boolean$ method enters a new symbol into the current scope on top of the stack.
The \texttt{name} parameter of type \texttt{String} must be matched with the identifier name in question.
The \texttt{type} parameter of type \texttt{Treenode.Type} must be matched with a valid AST-node type.
The method returns a boolean value indicating whether the operation was successful.

The $retrieveSymbol(name: String): Symbol$ method retrieves the symbol stored in the top sub-symbol table on the stack with the name indicated by the \texttt{name} parameter of type \texttt{String}.

The method on line 6 is concerned with testing whether a symbol is declared in the local scope.

The $declaredLocally(name: String): Boolean$ method tests whether name is present in the symbol table's current scope.

\textbf{Implementation}\\
The implementation of the $enterSymbol$ method is the most complex one in the symbol table. 
This is because when entering a new symbol it is relevant to make sure that the symbol doesn't already exist in the current scope.
Scopes are further explained in the scope section, section \ref{sec:scopeRules}.
In the case of a function, it is possible to enter multiple symbols with the same name, as long as they have different parameter-type signatures. 
Functions, however, have to have the same amount of parameters.
The implementation is shown in snippet \ref{lis:STEnterSymbol}.

\begin{lstlisting}[language=java,label=lis:STEnterSymbol,caption=Implementation of enterSymbol.]
override fun enterSymbol(name: String, type: AstNode.Type): EnterSymbolResult {
	val entry = symbolTable.last[name]
	return if (entry != null) {
		val entryFirst = entry.first()
			if (entryFirst is AstNode.Type.Func.ExplicitFunc && type is AstNode.Type.Func.ExplicitFunc) {
			checkFunctionIsAllowed(type, name).also { if (it == EnterSymbolResult.Success) entry.add(type) }
		} else EnterSymbolResult.IdentifierAlreadyDeclared
		} else {
			val res = if (type is AstNode.Type.Func.ExplicitFunc) checkFunctionIsAllowed(type, name)
			else EnterSymbolResult.Success
			res.also { if (it == EnterSymbolResult.Success) symbolTable.last[name] = mutableListOf(type) }
	}
}
\end{lstlisting}

\textbf{Considerations}\\
As described previously, scoping is handled with a stack data structure and the sub-symbol tables themselves are handled with a hash table data structure.
This is not the only way to implement symbol tables.
Instead of a stack data structure, the scope management could have been implemented using an attribute based data structure.

Specifically the depth of the scope could be implemented as attribute of the symbol itself.
By using indexing on the scope management data structure, the entire symbol table could have been implemented as one large table.
However, by simply comparing the amount of code necessary to implement the single-table and the stack table implementations, it was apparent that the stack table implementation would be simpler to implement.

The choice of using a hash table data structure for the sub-tables themselves was based upon the time complexity of the essential procedures.
As mentioned earlier, the $retrieveSymbol$ method would be called more frequently than the remaining methods.
It was therefore deemed sufficient to implement the data structure with the most efficient search operations.
Insertion or retrieval in a hash table data structure can be performed in constant time.
