%THIS SECTION IS SUPPOSED TO BE A SUBSECTION TO THE PARSER SECTION.
%Should present the interface, which all symbol table implementations must implement.
%Should clarify all considerations through the design process.
\subsection{Symbol Table}
\label{sec:symbolTable}
In order for the parser to handle symbols and their types, a symbol table is needed to store symbol information.
This allows the parser to do type checking.

With this in mind, an explanation of the symbol table is explained in this section. 
Firstly the considerations from which the symbol table was implemented, followed by an explaination of the implementation

\textbf{Considerations}\\
Scoping is handled with a stack data structure and the sub-symbol tables themselves are handled with a hash table data structure.
This is not the only way to implement symbol tables.
Instead of a stack data structure, the scope management could have been implemented using an attribute based data structure.

Specifically the depth of the scope could be implemented as attribute of the symbol itself.
By using indexing on the scope management data structure, the entire symbol table could have been implemented as one large table.
However, by simply comparing the amount of code necessary to implement the single-table and the stack table implementations, it was apparent that the stack table implementation would be simpler to implement.

The choice of using a hash table data structure for the sub-tables themselves was based upon the time complexity of the essential procedures.
Insertion or retrieval in a hash table data structure can be performed in constant time, however as the parser will look through multiple scopes, retrieval is only constant time in the context of a singular scope.
Retrieval is the method used most often, which means that the running time of the retrieval is most important, however that's not possible because of the above reasons. 
It is, however, not an major issue as it is not expected that the user of HCL are going to use deep nested scopes, and therefore the running time shouldn't be affected massively.

\textbf{Implementation}\\
The interface consists of five functions as seen in snippet \ref{lis:STInterface}.

\begin{lstlisting}[language=java,label=lis:STInterface,caption=The interface which all symbol table implementations must implement.]
interface ISymbolTable{
	fun openScope()
	fun closeScope()
	fun enterSymbol(name: String, type: TreeNode.Type): EnterSymbolResult
	fun retrieveSymbol(name: String): Symbol
	fun declaredLocally(name: String): Boolean
}
\end{lstlisting}
The interface, and implementations of it, are defined by opening and closing scopes as well as entering and retrieving symbols.

When opening a scope, a new scope is added to the list, while the top scope is popped when running the $closeScope()$ method.

\begin{lstlisting}[language=java,label=lis:STEnterSymbol,caption=A simplified version of the enterSymbol implementation.]
override fun enterSymbol(name: String, type: AstNode.Type): EnterSymbolResult {
	val entry = symbolTable.last[name]
	return if (entry != null) {
		if (entry.first() is ExplicitFunc && type is ExplicitFunc) {
			checkFunctionIsAllowed(type, name).also { if (it == EnterSymbolResult.Success) entry.add(type) }
		} else EnterSymbolResult.IdentifierAlreadyDeclared
	} else {
	// add new symbol
	....
	}
}
\end{lstlisting}

The implementation of the $enterSymbol$ method, as shown in snippet \ref{lis:STEnterSymbol}, is the most complex one in the symbol table. 
This is because when entering a new symbol it is relevant to make sure that the symbol doesn't already exist in the current scope.
Scopes are further explained in section \ref{sec:scopeRules}.
In the case of a function, it is possible to enter multiple symbols with the same name, as long as they have different parameter-type signatures. 
Functions, however, have to have the same amount of parameters.
