% !TeX root = ../main.tex
% This document needs to include:
% - An introduction for the concern.
% - Bases for section, specifically EBNF in previous section.
% - Difference between EBNF CFG.
% - HCL as CFL and HCL's CFG.
% - Proof that HCL is a CFL.
% - Proof that shows whether HCL is a regular language or not.
% - Description of identifiers and types in HCL using computational models or Regex.
% - Conclusion.
\section{Mathematical Syntax Theory}
This section contains the concise definitions of HCL's syntax.
First the proper CFG recognizing HCL is presented.
The CFG is used to prove that HCL is a Context-free Language (CFL). 
Afterwards the most specific computational model for describing HCL is identified and presented. 
Finally the token categories are described in a computational model. 
The conclusions of this section provides the basis for implementing and testing the parser of the HCL compiler. 
The majority of this section is based on the EBNF details touched upon in appendix \ref{AppendixEBNF}.

\subsection{HCL as an CFL}
This subsection investigates whether HCL is a CFL. 
To determine this, a Pushdown Automaton (PDA) is constructed.
The reason being that the set of CFL's is equivalent to the set of languages recognized by a PDA.
To construct the PDA the following rules are used:
\begin{center}
	\begin{itemize}
		\item The marker symbol \$ is first placed on the stack.
		\item Second, the following steps are repeated until all production rules have been handled.
		\begin{itemize}
			\item If the top of the stack is a variable symbol V. 
			The machine will nondeterministically select one of the right production rules of V and replace V with that.   
			\item If the top of stack is a terminal symbol t. 
			The next character on the input string is read and compared to t. 
			If matched, repeat. 
			If not, reject the branch.
			\item If the top of stack is the marker symbol \$, enter the accept state of the automaton. 
			This will accept the input string.
		\end{itemize}
	\end{itemize}
\end{center}

To create the automaton, a start state needs to be added at the beginning of the automaton. 
A transition rule on the form $\epsilon,\epsilon->S\$$ needs to be added between the start state and a new state called the loop state.
Here S is a variable.
From the loop state to the accept state there needs to be a transition rule on the form $\epsilon,\$->\epsilon$.
To convert the EBNF's production rules to a PDA, two conversion rules need to be followed.

For each production rule of the form $A->w$, where A is a variable and w is a string the transition rule $\epsilon,A->w$ needs to be added to the automaton from the loop state back to the loop state.
For each terminal a transition rule $a,a->\epsilon$ needs to be added to the automaton from the loop state back to the loop state.
The final non-deterministic PDA (NPDA) can be seen in Appendix \ref{NPDA} on page \pageref{NPDA}.
Thus concludes the proof.

\subsection{Regularity of HCL}
HCL is as CFL recognized by the CFG described in the previous subsections.
Therefore, it is now relevant to investigate whether HCL is a regular language. 
If that is the case, it is possible to create a Finite State Automaton (FSA) that recognizes the language.

Since the set of regular languages is a subset of the context free languages, this may be the case.
A FSA is more trivial to implement than a NPDA. 
Since the HCL language is a relatively complex language, it would be unrealistic to sequentially go through the entire language while trying to construct a FSA that recognizes HCL.

Considering the fact that the regular languages are closed under the union, concatenation, and the star operations, it would be sufficient to locate a subset of HCL that is not a regular language.
The transition rules described in the EBNF, state that a subset of HCL might consist of an arbitrary number of equal amounts of left and right parentheses.
This subset, referred to as $HCL_{subset}$ can be described as the set:
\begin{center}
	$HCL_{subset} = \{a^nbc^n | a \in \{(\} \wedge b \in P \wedge c \in \{)\}\}$
\end{center}
where P is an arbitrary set containing elements, that are legal in the context.

To prove that HCL is not a regular language, the pumping lemma is used.
The pumping lemma states that:
\begin{center}
	If A is a regular language, there exists a number $p>0$ so all strings $s \in A $, where $|s| \geq p$ can be divided $s = xyz$ so the following properties are true:
	\begin{itemize}
		\item $xy^iz \in A$ for $i \geq 0$
		\item $|y| > 0$
		\item $|xy| \leq p$
	\end{itemize}
\end{center}

First the division is chosen for s.
$s = xyz$, where $s = (((...b)))...$.
Meaning that x consists only of left parentheses, y consists of a concatenation of left parentheses and elements from P and z consists only of right parentheses.

The length of the string is $p < |s| < (p3)/2$, since $s = a^nbc^n$, where $n = p/2$ and $0 < |b| \leq p/2$.

This means that the third condition is upheld since $|xy|$ cannot be greater than p.
The second condition is also upheld since y consists of at least some left parentheses. 
If the string does contain right parentheses, which it does, y cannot be of the length 0.

The first condition is not upheld since if $i = 0$ the new string cannot contain an equal amount of left and right parentheses because of the same reason as to why the second condition is upheld.

This concludes the proof by contradiction.
Since HCL does not uphold the required property for it to be a regular language, it is not.
Therefore it cannot be described by a FSA, therefore the implementation will not be trivial.

\subsection{Types and Identifiers}
So far the syntax of HCL has been properly defined as a CFL. 
Meaning that all variables, terminals and the transition rules for recognizing HCL has been described in a computational model as seen in appendix \ref{NPDA} on page \pageref{NPDA}.
It is now relevant to define how the types and identifiers in HCL can be described, or recognized, by a computational model.
The goal is to formally describe the languages $L_{none}$, $L_{bool}$, $L_{number}$, $L_{text}$ which are the type languages and the identifier language $L_{identifiers}$.

An alphabet is a finite set of symbols, a string is a finite sequence of symbols and a language is a set of strings.
To describe the languages mentioned above, the Alphabets used in the informal and formal notations needs to be defined. 
These can be seen in Table 3.1.

\begin{table}[!htb]
	\centering
	\begin{tabular}{|l|l|}
		\hline
		\textbf{Alphabet} & \textbf{Set}                            \\ \hline
		T                 & \{True\}                                \\ \hline
		F                 & \{False\}                               \\ \hline
		$Z^+$             & The set of all positive integers        \\ \hline
		$Z^-$             & The set of all negative integers        \\ \hline
		R                 & The set of all real numbers             \\ \hline
		D                 & \{$Z^+ \cup Z^-$\}                      \\ \hline
		$S_p$             & The set of all special characters      \\ \hline
		$L_l$             & The set of all lower-case letters       \\ \hline
		$L_u$             & The set of all upper-case letters       \\ \hline
		C                 & $S_p \cup L_l \cup L_u \cup D$          \\ \hline
		$R_{eserved}$     & The set of all reserved keywords, specialchars and types \\ \hline
	\end{tabular}
	\caption{Table containing all utility alphabets used to formally describe the type and identifier languages}
\end{table}

By using the set construction method, the languages can now be defined as shown in Table 3.2.

\begin{table}[!htb]
	\centering
	\begin{tabular}{|l|l|}
		\hline
		\textbf{Name}     & \textbf{Set}                                    \\ \hline
		$L_{none}$        & \{w | w $\in$ \{None\}\}                        \\ \hline
		$L_{bool}$        & \{w | w $\in$ (T $\cup$ F)\}                    \\ \hline
		$L_{number}$      & \{w | w $\in$ R\}                               \\ \hline
		$L_{text}$        & \{"w" | all strings from arbitrary alphabets\}  \\ \hline
		$L_{identifiers}$ & \{w | w $\in (L_{text} - \{", "\}) \wedge w \notin R_{eserved}$\} \\ \hline
	\end{tabular}
	\caption{Table containing the type and identifier alphabets described by the set builder construction}
\end{table}

Intuitively, it is trivial that all of the languages defined above are regular languages, except for $L_{identifiers}$ since it would need to make sure that no strings are an element of $S_p$.
This is the case since a regular language cannot keep track of more than one symbol in terms of counting and iterations.

Since all of the languages in Table 3.2(except for $L_{identifiers}$) consists only of arbitrary concatenations of elements of various alphabets, this means that they are indeed regular. 

Now that it has been determined that the languages are regular, a computational model can now be chosen to describe them more specifically.

For the sake of convenience, $R^+$ will be used at a shorthand for $RR^*$ \footnote{Where * is Kleene star}, meaning that $R^+ \cup \epsilon = R^*$ since $R^*$ consists of all strings that are zero or more concatenations from R, and $RR^*$ consists of all strings that are one or more concatenations of strings from R.
The resulting regular expressions are shown in Table 3.3.

\begin{table}[!htb]
	\centering
	\label{my-label}
	\begin{tabular}{|l|l|}
		\hline
		\textbf{Name}     & \textbf{Regular Expression}         \\ \hline
		$L_{none}$        & none                                \\ \hline
		$L_{bool}$        & True $\cup$ False                   \\ \hline
		$L_{number}$      & $D.(D^+)^*$                       \\ \hline
		$L_{text}$        & $"C^*"$                               \\ \hline
	\end{tabular}
	\caption{Regular Expressions describing the type and identifier languages}
\end{table}

In conclusion HCL is a context free language, meaning that it can be described by a NDPA.
Some subsets of HCL is non-regular, meaning that HCL cannot be a regular language.
The token category type languages are all regular languages. 
These can therefore be described using regular expressions as presented in this section.



