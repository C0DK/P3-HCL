\section{Language Specification}
This section will describe the language designed for this project together with an explanation of which functionality and features the language will include.

During the analysis face of the project it was decided that the language designed would be used as a preliminary introduction to programming.
The language designed should therefore reflect this decision by being useful and understandable by users who are not advanced in programming and programming fundamentals.
This decision causes both restrictions in designing the language, but also gives the freedom of designing a language for users who has no biased opinions on already established languages. 

It was decided that the syntax of the language should reflect the spoken English language to some extend.
The idea of drawing inspiration from the spoken English language is supposed to give the users of the language an easier time expressing what they want the code to do.
In conjunction to designed after the spoken English language and the interest in utilizing high-order functions, the language design also draws inspiration from the functional programming paradigm.

Since the language is supposed to be used in correlation with the Arduino platform, it is nessecary to take the Arduino's \textbf{\textit{setup}} and \textbf{\textit{loop}} functions into account.
In order to follow this convention, a method called \textbf{\textit{loop}}, which takes a function body as a parameter, is established. 
The code within this body is mapped to the Arduino's \textbf{\textit{loop}} method.
Everything outside of the loop method, expressed as the "outer scope", is subsequently mapped to the Arduino's \textbf{\textit{setup}} method.

Following is a rundown of the features and conventions of the languages syntax that is planned for implementation:

\textbf{Types}\\
The language include only three primitive types, namely numbers (\textbf{\textit{num}}), text-strings (\textbf{\textit{txt}}) and the boolean type (\textbf{\textit{bool}}).\\
numbers are represented using double-precision floating-point format, also known as the \textbf{\textit{double}} type.
Booleans are represented using \textbf{true} or \textbf{false}.

The collection types of the language are \textbf{\textit{toubles}} and \textbf{\textit{lists}} where:\\
\textbf{\textit{toubles}} are collections of multiple types and \textbf{\textit{lists}} are collections of a single type.
Dictionaries that functions as a collection of key-value pairs are deemed NTH and are not planned for implementation. 

The language also includes the type \textbf{\textit{func}} which is a rename of the \textbf{\textit{lamda}} type.
Also included is the type \textbf{\textit{none}} which is a rename of the \textbf{\textit{void}} type. 

\textbf{Type Declaration}\\
Types can be declared in the language in different ways.
Firstly, types can be declared using the keyword \textbf{\textit{var}} (short for \textbf{variable\textit{}}) prefixed an assignment(line 1).
Declarations can also be explicit by prefixing the type of the variable before an assignment(line 2), or by prefixing the type before the variable without an assignment(line 3).
The three methods of declaration are seen in snippet below.
\begin{lstlisting}
var x = 5
txt s = "string"
bool b
\end{lstlisting}

Using the prefix \textbf{\textit{var}} enables the use of type-inference, where the type of the variable is automatically set during compile time \cite{typeinf}.
Types in the language are type-static, meaning that once a type is set, it cannot be overwritten without an explicit valid type cast.

\textbf{Method Declaration}\\
The type of a methods parameters as well as its return type are explicitly declared when a method is declared. 
The declaration is structured as an assignment where the methods name is declared on the left-hand-side prefixed by either \textbf{\textit{var}} or \textbf{\textit{func}}. 
The parameters of the method is declared on the right-hand-side inside parenthesis followed by an arrow (\textbf{\textit{->}}) expressing the return type. 

An example method that adds two numbers together and prints the value is shown below:
\begin{lstlisting}
var addNumbers = (num a, num b) -> none {
	a + b print
}
\end{lstlisting} 
A more explicit declaration of the same method can be seen in the snippet below:
\begin{lstlisting}
func<num,num,none> addNumbers = (num a, num b) -> none {
	a + b print
}
\end{lstlisting}

\textbf{Method Call}\\
In order to better reflect the spoken English language, method calls are structured in a way where the first parameter to a method is written before the method-call itself.
All subsequent parameters are written after the method-call and are delimited by the use of white-spacing and can be optionally grouped inside parenthesis.
When methods are passed in as parameters, they shall always be prefixed with '::' characters to differentiate between methods as parameters and other method calls. 
Lambdas can be used as parameters by encapsulation with curly-brackets. 

Below is an example of method calls where the program prints a name from a list:
The \textbf{where} method in this example returns an entry in the list based on a lambda expression.
\begin{lstlisting}
["john", "frank"] where {value is "frank"} print
\end{lstlisting}

\textbf{Loops}\\
Loops in the language are structured around iterations on lists.
Loops are created using either of the predefined methods \textbf{\textit{foreach}} or \textbf{\textit{do}} that takes a \textbf{\textit{list}} as first parameter and \textbf{\textit{func}} as second parameter.
When creating ranges the predefined \textbf{\textit{to}} method, that returns a list, can be used. 

Following is a small example that prints numbers using the \textbf{\textit{do}} method:
\begin{lstlisting}
1 to 10 do {value print}
\end{lstlisting}
There are no \textbf{\textit{while}} loop structures in the language.

\textbf{Operators}\\
Mostly all operators in the language are used as functions.
This means that all arithmetic operators are infix-functions that takes two numbers as parameters. 
The only operators that remains operators are the assign-operator (\textbf{=}) and the encapsulation-operators, namely parenthesis, square-brackets and curly-brackets.

Not having operators for arithmetic operations also means that there are no native operator precedence in the language.
This can be amended by structuring calculations in a left-to-right manner, or by encapsulating operations with parenthesis. 