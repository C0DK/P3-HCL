% !TeX root = ../main.tex
\section{Syntax Specification}
This section describes HCL's syntax.
It also explains the functionality and features of HCL.

HCL is designed to be used as a preliminary introduction to programming.
HCL reflects this design choice by being approachable and easy to understand by novices.

In order to make HCL accessible, its syntax reflects the English language to some extent.
The idea of drawing inspiration from the English language is supposed to give the users of HCL an easier time expressing what they want the code to do, giving HCL an almost pseudo-code like syntax.
In conjunction with designing according to the English language and utilizing high-order functions, HCL's design also draws inspiration from the functional programming paradigm.
In an effort to accommodate these ideas, almost all operations in HCL are done using functions. 

The following is a run-down of the features and conventions of HCL's syntax.

\subsection{Types}
HCL includes only three primitive types, numbers (\textbf{\textit{num}}), strings (\textbf{\textit{txt}}) and the boolean type (\textbf{\textit{bool}}).\\
Numbers are represented using double-precision floating-point format, also known as the \textbf{\textit{double}} type.
Booleans are represented using \textbf{true} or \textbf{false}.

The collection types of HCL are \textbf{\textit{tuples}} and \textbf{\textit{lists}} where, 
\textbf{\textit{tuples}} are structures composed of multiple types and \textbf{\textit{lists}} are collections of a single type.
While dictionaries are a staple in modern languages, they are not deemed necessary for the scope of this "first specification" of HCL.

HCL also includes the type \textbf{\textit{func}} which in some languages, particularly C\#, is known as the \textbf{\textit{lambda}} type.
Also included is the type \textbf{\textit{none}} which in some languages, particularly C\# and Java, is known as the \textbf{\textit{void}} type. 

Table \ref{tbl:types} lists all types in HCL and their \texttt{C++} equivalent.
\begin{table}[H]
	\centering
	\caption{Types in HCL}
	\label{tbl:types}
	\begin{tabular}{|c|c|}
		\hline
		HCL Type & C++ Equivalent \\ \hline
		num      & double         \\ \hline
		txt      & string         \\ \hline
		bool     & bool           \\ \hline
		tuple    & struct         \\ \hline
		list     & list           \\ \hline
		func     & lambda         \\ \hline
		none     & void           \\ \hline
	\end{tabular}
\end{table}
\textbf{Type Declaration}\\
Types can be declared in a few different ways, as the preliminary interviews indicate that users might be interested in declaring types both explicitly and implicitly.
Therefore, types can be declared using the keyword \textbf{\textit{var}} (short for \textbf{variable\textit{}}) prefixed to an assignment, listing \ref{lis:typedcls}, line 1.
Declarations can also be explicit by prefixing the type of the variable before an assignment, listing \ref{lis:typedcls}, line 2, or by prefixing the type before the variable without an assignment, listing \ref{lis:typedcls}, line 3.

Declaration of the different types are done in the following way:
\begin{lstlisting}[language=HCL,label=lis:typedcls,firstnumber=1]
num x = 10
txt y = "string"
bool z = true
list[num] w = [0,1,1.2]
tuple[txt,num] v = ("hello",2)
func[txt] u = ():txt { "hello" }
\end{lstlisting}

As previously stated, declaration can be done either with an implicit type and an assignment, explicit type and an assignment or explicit type.
This is shown through the \textbf{\textit{num}} type, in the following example:

\begin{lstlisting}[language=HCL,firstnumber=1]
var x = 5
num y = 10
num z

\end{lstlisting}

Using the prefix \textbf{\textit{var}} enables the use of type-inference, where the type of the variable is automatically inferred at compile time \cite{typeinf}.
HCL uses static typing, meaning that type-checking is done at compile time.

\subsection{Operators}
All arithmetic operators are defined as built-in functions in HCL.
This means that all arithmetic operators are infix-functions, which have expressions as parameters. 
The only operators that remain operators are the assign-operator and the encapsulation-operators, namely parenthesis, square-brackets and curly-brackets.

Not having operators for arithmetic operations also means that there is no native operator precedence in HCL.
This can be amended by structuring calculations in a left-to-right manner, or by encapsulating operations with parentheses, since parentheses have higher precedence than function calls. 
This workaround is shown below.

\begin{lstlisting}[language=HCL,firstnumber=1]
num x = 10 + 5 * 10 #Equals 150 in HCL
num y = 10 + (5 * 10) #Equals 60 in HCL, which corresponds with algebra.
\end{lstlisting}

\subsection{Functions}
\textbf{Function Declaration}\\
The type of a function's parameters, as well as its return type, are explicitly declared when a function is defined. 
The declaration is structured as an assignment, where the functions name is declared on the left-hand-side, prefixed by either \textbf{\textit{var}} or \textbf{\textit{func}}. 
The parameters of the function are declared on the right-hand-side, inside parentheses, followed by a colon, which declares the return type. 

An example function that adds two numbers together and returns the value is shown below:
\begin{lstlisting}[language=HCL,label=lis:addNumbers,firstnumber=1]
var addNumbers = (num a, num b): num {
    a + b
}
\end{lstlisting} 
A more explicit declaration of the same function can be seen in the snippet below:
\begin{lstlisting}[language=HCL,label=lis:hclExplicitTypeDcls,firstnumber=1]
func[num,num,num] addNumbers = (num a, num b): num {
	return a + b
}
\end{lstlisting}

The return statement is noticeably missing in the first function.
\texttt{return} is optional if the function body only contains a single expression. 
This makes sense when using lambdas, where return statements are often implicit, in modern programming languages. 
If the function contains more than one expression \texttt{return} has to be explicit.

\textbf{Grammar for declarations}\\
The grammar for declarations in HCL (specified in Appendix \ref{AppendixEBNF}) acknowledges that type declarations and function declarations are the same by describing them with the same non-terminal, \texttt{<Dcl>}.
\begin{align*}
	\texttt{<Dcl>}\to & \texttt{ <ImplicitType> identifier [equals <DclValue>]}
\end{align*}
The \texttt{<ImplicitType>} non-terminal, unlike the <Type> non-terminal, refers to all type productions, including the \texttt{func} production that allows programmers to declare functions without having to specify the parameter and return types.

The \texttt{<DclValue>} non-terminal refers to both \texttt{<Expr>} (expression) for type declarations and \texttt{<LambdaExpr>} (function parameters and body) for function declarations.

The below listing showcases declarations in HCL using type inference as described above.
The listing includes both assignment of expressions and assignment of lambda-expressions.

\begin{lstlisting}[language=HCL,caption={Implicit type declarations in HCL.
},firstnumber=1]
var x = 5
var y = 3
var z = x > y

var w = x + y

var v = (num v, num u): num { v * u}
\end{lstlisting}

\textbf{Function Calls}\\
In order to better reflect the English language, function calls are structured in a way where the first parameter to a function is written before the function call itself.
This first parameter will be referred to as the prefix parameter. 
All subsequent parameters are written after the function call and are delimited by the use of white-spacing and can be optionally grouped inside parentheses.
These subsequent parameters will be referred to as the postfix parameters.
Lambdas can be used as parameters by encapsulation with curly-brackets. 

Below is an example of function calls where the program prints a name from a list:
The \textbf{where} function in this example returns an entry in the list based on a lambda expression.
\begin{lstlisting}[language=HCL,label=lis:hclPrintnFromList,firstnumber=1]
["john", "frank"] where {value is "frank"} print
\end{lstlisting}

\textbf{Grammar for function calls}\\
Function calls are done with infix notation, meaning that if a given function is defined with parameters, the first argument has to appear before the function identifier, while all remaining arguments must appear after, in sequence.

The following listing showcases this.
\begin{lstlisting}[language=HCL,caption={Function call notation in HCL.
printHelloWolrd, print and printWords are example functions.
},firstnumber=1]
printHelloWorld

"Hello World !!" print

"Hello" printWords "World" "!!"
\end{lstlisting}

With this in mind, the grammar for function calls is specified as follows:
\begin{align*}
	\texttt{<FunctionCall>}\to & \texttt{ identifier}\\
	| & \texttt{ <FunCallParam> identifier [<FunCallParams>]}
\end{align*}
As it is not allowed to have parameters on the right-hand side of the identifier without also having one on the left-hand side, this rule cannot accurately be described with only one production.

The first production shows how the function is called if it has no parameters.
The second production shows that if the function has parameters, there must always be exactly one parameter on the left-hand side of the identifier, and optionally an ordered following of parameters on the right-hand side.

The non-terminal \texttt{<FunCallParam>} refers to any legal parameter for function calls, including identifiers, literals, lambda expressions, and other function calls.

The non-terminal \texttt{<FunCallParams>} refers to at least one \texttt{<FunCallParam>}, optionally enclosed in parentheses.

\textbf{Loops}\\
Loops in HCL are structured around iterations on lists.
Loops are created using either of the predefined functions \textbf{\textit{foreach}} or \textbf{\textit{do}} that takes a \textbf{\textit{list}} as its first parameter and \textbf{\textit{func}} as its second parameter.
When creating ranges the predefined \textbf{\textit{to}} function, that returns a list, can be used. 

\textbf{High order functions}\\
As mentioned in section \ref{subsec:hofhcl}, HCL includes high-order functionality. 
To declare that a function will accept other functions, the developer should simply declare a parameter of type \textit{func} with the expected input and output types explicitly stated from the function.

Both predeclared functions and anonymous in-line functions can be passed as parameters. 
However, when using a predeclared function, the identifier should be prefixed with a "colon" (\textit{\textbf{:}}) to indicate that the function should be passed and not invoked. 
Since lambdas should not be prefixed with a colon, it makes it impossible for a lambda to be invoked in-line. 
Therefore lambdas may only be passed as parameters, and never invoked by themselves.
Below is an example of how high-order functionality is used in HCL.
\begin{lstlisting}[language=HCL,label=lis:hclTypeDcls,firstnumber=1]
var modifyNumber = (num n, func[num, num] modifyFunction): num { n modifyFunction }

# Used with lamdas
5 modifyNumber { value * 2 } # 10
5 modifyNumber { value - 3 } # 2

# Used with predeclared functions
var times2 = (num v): num { v * 2 }

8 modifyNumber :times2 # 16

\end{lstlisting}

\textbf{Overloading}\\
HCL supports function overloading. 
This means that multiple functions can share the same name and have different implementations based on the different parameters that the function accepts.

An example could be the plus function. 
The plus function could accept two numbers or two strings, but the implementation of the plus function would depend on the type.
If two numbers were passed, the obvious result is the sum of the two numbers.
If, on the other hand, two strings were passed, the function could be implemented in such a way that the result would be the concatenation of the two strings.

Function overloading in HCL is limited in that functions can not be overloaded with the same parameters but different return types. 
Functions can only be overloaded to return different types, if the types of the parameters vary.
Furthermore, overloading with different amounts of parameters is not possible in HCL.
This is because of multiple reasons, one being simplicity for the user, not accidentally adding an unexpected parameter to a function call. 
Another being simplicity in the parsing of the function parameters. 
Since the ability to overload with different amount of parameters brings next to nothing to HCL, it will not be implemented.
Should it be necessary to overload functions with varying lengths of input-parameters, it can be done utilizing tuples or lists.

Function overloading requires nothing special of the user, simply declare two functions with the same amount of parameters, but with different types. 
See listings below for example

\begin{lstlisting}[language=HCL,label=lis:hclEqualsThenElse,firstnumber=1]
var isAwesome = (txt name): bool {
	name equals "HCL" thenElse {true} {false}
}
var isAwesome = (num number): bool {
	number equals 42 thenElse {true} {false}
}

42 isAwesome # true
"Java" isAwesome # false
\end{lstlisting}

The main reasoning for including function overloading is simplicity for the user.
For instance, printing; the print function should be able to print all basic types, so that the user does not have to worry about the type of what he is about to print.

\subsection{Arduino functions}
Since HCL is supposed to be used on the Arduino platform, it is necessary to take the Arduino's \textbf{\textit{setup}} and \textbf{\textit{loop}} functions into account. 
These are explained in section \ref{arduinoC}.
In order to follow this convention, a function called \textbf{\textit{loop}}, which takes a function body as a parameter, is therefore defined in HCL.
The \textbf{\textit{loop}} function iterates consecutively.
Everything outside of the loop function, expressed as the "outer scope", is subsequently mapped to the \texttt{C++} global scope, if it is a declaration.
Otherwise it is mapped to the Arduino's \textbf{\texttt{Setup}} function.

This can be seen through the following example in HCL:
\begin{lstlisting}[language=HCL,caption={Example of an arduino program in HCL},label={lis:hclArduinoExample},firstnumber=1]
var led1 = 15 
led1 LedSetup OUTPUT 

"Hello" print

{
	led1 setDigitalPin true
	1000 delay
	led1 setDigitalPin false
	1000 delay
} loop

"World" print 
\end{lstlisting}
The sourcecode in HCL, in snippet \ref{lis:hclArduinoExample} would translate into the following \texttt{arduino-C++} code:
\begin{lstlisting}[language=C,label={lis:cppArduinoExample},firstnumber=1]
int led1 = 13;

void setup() {
	pinMode(led1, 1);
	print("Hello");
	print("World");
}

void loop() {
	digitalWrite(led1, 1);
	delay(1000);
	digitalWrite(led1, 0);
	delay(1000);
} 
\end{lstlisting}

\subsection{Generics}
HCL includes simple generics.
Generics are used to add abstraction to types, and handle the type generically.
With HCL being a language with a great focus on high order functions, this functionality becomes a must have.
This is due to some of the fundamental higher-order functions, like the map function.

This section will use the map function, as mentioned in section \ref{txt:mapHorder}, to exemplify how generics are used in HCL

In theory this problem could be solved with overloading. 
However, this would require overloading of a lot of functions with all the possible types within HCL. 
Since tuples are distinguished by the types within it, there are a theoretical infinite amount of types within HCL, which in turn makes overloading for this case impossible.

To get a better grasp of how generics are used in HCL, consider the following implementation of the map function:
\begin{lstlisting}[language=HCL,label=lis:hclMapping,firstnumber=1]
var map = (list[T1] elementsToMap, func[T1, T2] mapFunction): list[T2] {
	list[T2] mappedValues = []
	elementsToMap onEach {mappedValues add (value mapFunction)}
	return mappedValues
}
\end{lstlisting}

Generic types in the above example are T1 and T2. 
Since nothing type specific ever happens with these types, they can all be handled the same way, whether they are strings, numbers or any other type.
