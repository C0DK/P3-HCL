\section{Language Specification}
This section will describe the language designed for this project together with an explanation of the functionality and features the language will include.

During the analysis phase of the project, it was decided that the language, would be designed to be used as a preliminary introduction to programming.
The language designed should therefore reflect this decision, by being useful, approachable and easy to understand for users that might not know programming and programming fundamentals.
This decision puts restrictions on the design of the language, however it also gives a degree of freedom in the design, as the users have no previous experience with other languages.


It was decided that the syntax of the language should reflect the spoken English language to some extend.
The idea of drawing inspiration from the spoken English language is supposed to give the users of the language an easier time expressing what they want the code to do, giving the language an almost "pseudo-code-like" syntax.
In conjunction to designing after the spoken English language and the interest in utilizing high-order functions, the language design also draws inspiration from the functional programming paradigm.
In an effort to accommodate these ideas and inspirations it is decided that almost all operations in the language will be done using functions. 

Since the language is supposed to be used in correlation with the Arduino platform, it is necessary to take the Arduino's \textbf{\textit{setup}} and \textbf{\textit{loop}} functions into account, which are explained in section \ref{arduinoC}.
In order to follow this convention, a function called \textbf{\textit{loop}}, which takes a function body as a parameter, is established. 
The code within this body is mapped to the Arduino's \textbf{\textit{loop}} function.
Everything outside of the loop function, expressed as the "outer scope", is subsequently mapped to the Arduino's \textbf{\textit{setup}} function.

The following is a rundown of the features and conventions of the language's syntax, that is planned for implementation:

\textbf{Types}\\
The language includes only three primitive types, numbers (\textbf{\textit{num}}), text-strings (\textbf{\textit{txt}}) and the boolean type (\textbf{\textit{bool}}).\\
Numbers are represented using double-precision floating-point format, also known as the \textbf{\textit{double}} type.
Booleans are represented using \textbf{true} or \textbf{false}.

The collection types of the language are \textbf{\textit{toubles}} and \textbf{\textit{lists}} where:\\
\textbf{\textit{tubles}} are collections of multiple types and \textbf{\textit{lists}} are collections of a single type.
Dictionaries, which are collections of key-value pairs, are deemed NTH and are not planned for implementation. 

The language also includes the type \textbf{\textit{func}} which in some languages, particularly C\#, is known as the \textbf{\textit{lamda}} type.
Also included is the type \textbf{\textit{none}} which in some languages, particularly C\# and java, is known as the \textbf{\textit{void}} type. 

\textbf{Type Declaration}\\
Types can be declared in a few different ways, as it was found during the preliminary interviews that users might be interested in declaring types both explicitly and implicitly.
Therefore, types can be declared using the keyword \textbf{\textit{var}} (short for \textbf{variable\textit{}}) prefixed to an assignment(line 1).
Declarations can also be explicit by prefixing the type of the variable before an assignment(line 2), or by prefixing the type before the variable without an assignment(line 3).
The three methods of declaration are seen in snippet below.
A  \textbf{\textit{num}} example is also shown.
\begin{lstlisting}
var x = 5
num y = 10
txt s = "string"
bool b
\end{lstlisting}

Using the prefix \textbf{\textit{var}} enables the use of type-inference, where the type of the variable is automatically inferred during compile time \cite{typeinf}.
The language uses static typing, meaning that type-checking is done at compile time.

\textbf{Function Declaration}\\
The type of a functions parameters, as well as its return type, are explicitly declared when a function is defined. 
The declaration is structured as an assignment, where the functions name is declared on the left-hand-side, prefixed by either \textbf{\textit{var}} or \textbf{\textit{func}}. 
The parameters of the function is declared on the right-hand-side inside parentheses, followed by an arrow (\textbf{\textit{->}}), which declares the return type. 

An example function that adds two numbers together and prints the value is shown below:
\begin{lstlisting}
var addNumbers = (num a, num b) -> none {
	a + b print
}
\end{lstlisting} 
A more explicit declaration of the same function can be seen in the snippet below:
\begin{lstlisting}
func<num,num,none> addNumbers = (num a, num b) -> none {
	a + b print
}
\end{lstlisting}

\textbf{Function Calls}\\
In order to better reflect the spoken English language, function calls are structured in a way where the first parameter to a function is written before the function call itself.
All subsequent parameters are written after the function call and are delimited by the use of white-spacing and can be optionally grouped inside parentheses.
When functions are passed in as parameters, they will be prefixed with '::' characters to differentiate between functions as parameters and function invoking. 
Lambdas can be used as parameters by encapsulation with curly-brackets. 

Below is an example of function calls where the program prints a name from a list:
The \textbf{where} function in this example returns an entry in the list based on a lambda expression.
\begin{lstlisting}
["john", "frank"] where {value is "frank"} print
\end{lstlisting}

\textbf{Loops}\\
Loops in the language are structured around iterations on lists.
Loops are created using either of the predefined functions \textbf{\textit{foreach}} or \textbf{\textit{do}} that takes a \textbf{\textit{list}} as its first parameter and \textbf{\textit{func}} as its second parameter.
When creating ranges the predefined \textbf{\textit{to}} function, that returns a list, can be used. 

Following is a small example that prints numbers using the \textbf{\textit{do}} function:
\begin{lstlisting}
1 to 10 do {value print}
\end{lstlisting}
There are no equivalent functions to the \textbf{\textit{while}} loop structure in the language, as all looping will be done using iterations.

\textbf{Operators}\\
Most of the operators in the language are defined as functions.
This means that all arithmetic operators are infix-functions that takes two numbers as parameters. 
The only operators that remains operators are the assign-operator (\textbf{=}) and the encapsulation-operators, namely parenthesis, square-brackets and curly-brackets.

Not having operators for arithmetic operations also means that there are no native operator precedence in the language.
This can be amended by structuring calculations in a left-to-right manner, or by encapsulating operations with parenthesis. 