%Describe CFG in HCL
%The commented-out section is the section deemed as extraneous for the report.
\section{Context-free Grammar of HCL}
\label{CFGdescription}
In the following section, a method to describe the production rules of the CFG for HCL will be chosen.
Two notational methods for describing production rules are typically used, Backus-Naur Form and Extended Backus-Naur form. 
Both of these will be touched upon in this section.

\textbf{Backus-Naur Form} (BNF) is a method for visualizing the production rules of a Context-free grammar (CFG).
Below is an example of such a production rule.
\begin{align*}
	\texttt{<Sentence>}\to & \texttt{ <StringOfWords> \$}\\
	\texttt{<StringOfWords>}\to & \texttt{ word <StringOfWords>}\\
	| & \texttt{ }\lambda
\end{align*}

All non-terminals are presented using PascalCase in angle brackets.
Terminals are presented using camelCase.
Non-terminals can be rewritten to one of the strings on right-hand side of the production rule.
A "|" symbol indicates that there is more than one production rule on the right-hand side.

\textbf{Extended BNF} (EBNF) is another way to visualize the production rules of a CFG.
It is designed to be more concise than BNF.
In BNF, if a non-terminal is structured in two or more different ways, each way has to be noted with a new rule.
In EBNF the number of rules can be reduced.
This is done by marking iterative and optional arguments, as well as choice of arguments, within that one rule.
\cite{SebestaEBNF}

The notation for EBNF is as follows:
\begin{align*}
	\text{Repetition (zero or more):} & \texttt{ \{arg\}}\\
	\text{Repetition (one or more):} & \texttt{ \{arg\}+}\\
	\text{Optional (zero or one):} & \texttt{ [arg]}\\
	\text{Choice:} & \texttt{ (arg1 | arg2)}
\end{align*}

The grammar for sentences can be written in EBNF as:
\begin{align*}
	\texttt{<Sentence>}\to & \texttt{ \{word\} \$}
\end{align*}

The grammar of HCL has been described using EBNF, and has been included in the appendix (Appendix \ref{AppendixEBNF}).
