%This section covers the implementation of the lexer
\section{Lexer implementation}
As described in section \ref{LexerChoice}, it was decided that the lexer (and therefore the parser), would be written by hand.
As mentioned in section \ref{REFTOPROGRAMLANGUAGECHOICE}, the compiler for HCL will be written in kotlin.
Below is an explanation of the implementation of the lexer.

The lexer is comprised of 3 different classes: the token class, the positional token class and the lexer class.

The token class simply contains the type of the token. 
It is implemented as a Sealed class. 
A sealed class in Kotlin, simply specifies that a value can have one specific type, from a predefined set\cite{KotlinSealed}.
It is essentially an extended enum.
\begin{lstlisting}[language=java,label=lis:tokenClass,caption=A snippet from the token class .,firstnumber=9]
sealed class Literal : Token() {
	class Text(val value: String) : Literal() {
	override fun toString() = super.toString() + "[$value]"
	}
	class Number(strValue: String, val value: Double = strValue.toDouble()) : Literal() {
	override fun toString() = super.toString() + "[$value]"
	}
	class Bool(val value: Boolean) : Literal() {
	override fun toString() = super.toString() + "[$value]"
	}
}
\end{lstlisting}

The positional token class is a wrapper class from a token, and simply contains additional information regarding linenumber and line index.
The positional token is mainly implemented to help with error reporting.
\begin{lstlisting}[language=java,label=lis:PositionalTokenClass,caption=A snippet from the token class .,firstnumber=10]
class PositionalToken(val token: Token, val lineNumber: Int, val lineIndex: Int)
\end{lstlisting}

The bulk of the logic regarding the lexer is done in the lexer class.
The lexer class implements the ILexer interface.

\begin{lstlisting}[language=java,label=lis:Lexer,caption=The Lexer .,firstnumber=8]
class Lexer(private val inputContent: String) : ILexer {
\end{lstlisting}

The ILexer defines a getTokenSequence function, which will give the token sequence and an inputLine function.
The getTokenSequence function will split the text string.
\begin{lstlisting}[language=java,label=lis:LexerStringSplit,caption=The string is being split and new lines are added to the string .,firstnumber=8]
override fun getTokenSequence(): Sequence<PositionalToken> = buildSequence {
	val currentString = StringBuilder()
	inputContent.split(endOfLineRegex).forEachIndexed lineIterator@{ lineNumber, line ->
		(line + '\n').forEachIndexed { indexNumber, char ->
			if (char !in listOf(' ', '\t')) currentString.append(char)
			if (char == '#') {
				// rest of line is comment
				...
			}
	with(currentString.toString()) {
\end{lstlisting}		
It will then match the characters to either a type, a boolean value, a special character, an identifier or a number literal.
If this is not possible, it can be assumed that the split string, meaning the current line, is done.
\begin{lstlisting}[language=java,label=lis:LexerStringMatcher,caption=The split string is being matched to a token,firstnumber=19]

when {
	// special char
	char.isSpecialChar() -> char.getSpecialCharTokenOrNull()
	isNextSpecialCharWhiteSpaceOrComment(lineNumber, indexNumber) -> when (this) {
		// types
		"var" -> Token.Type.Var()
		"none" -> Token.Type.None()
		"txt" -> Token.Type.Text()
		"num" -> Token.Type.Number()
		"bool" -> Token.Type.Bool()
		"tuple" -> Token.Type.Tuple()
		"list" -> Token.Type.List()
		"func" -> Token.Type.Func()
		//bool
		"true" -> Token.Literal.Bool(true)
		"false" -> Token.Literal.Bool(false)
		"return" -> Token.Return()
		// identifier or number literal
		else -> getLiteralOrIdentifier()
	}
	else -> null
}
\end{lstlisting}
