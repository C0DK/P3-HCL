\section{Parser implementation}

Since the compiler language kotlin is completely interoperable with java, both SableCC and ANLTR could be obvious choices for parser if the group decided to go with a generated parser. 
However, for the parser of HCL it was decided that since HCL's CFG is LL(k), the parser should be handwritten. 
This is partly due to the control that a handwritten parser offers, but primarily out of curiosity and for educational purposes.

Due to the nature of HCL, the parser can be implemented as a single pass parser. 
This is because the user may not access any variables that has not yet been declared. 
Furthermore as mentioned HCL is a LL(K) language, and it was decided that the parser should be a LL(K) recursive decent parser.

Both syntactical and contextual analysis is done within the parser. 
This means that there is no additional post parser type checking or contextual analysis step. 
One major reason for doing in line contextual analysis is the way functions are invoked in HCL. 
The only way to know whether anything is a function call, is by analyzing the types. 
This means that in order to create a proper abstract syntax tree the type checking must be done in line. 
It could also potentially be done afterwards, but that would require quite a big amount of post processing, that would be a lot more complex, than just doing type checking within the parser. 
This also means that the outputted abstract syntax tree is in fact already a decorated syntax tree. 
For instance function calls already has information on its return type, even though the function may have multiple return types based on the overloading of the function.

\subsection{Implementing a recursive decent parser}

A LL(K) recursive decent parser will do look aheads until it is aware what branch should be parsed.
This can be figured out from the first and follow sets, that can be produced from the CFG.
Once it figures out which non-terminal to parse it will call a parse method that should parse that part.
The parse method can then accept terminals or parse further non-terminals.
It should be the responsibility of the accept method to advance the lexical token stream.

Lets have a quick and concrete example from HCL. 

The line to parse is "num x = 5" The token stream to parse will look as follows:

\begin{itemize}
	\item Number type
	\item Identifier = x
	\item Equals
	\item Number Literal = 5
	\item End of line
\end{itemize}

The parser will see that the first token is a number type. 
Since only declarations have types in its first set, the parser now knows that it should parse a declaration.

The EBNF reveals that a declaration is:
\begin{itemize}
	\item Non-terminal Type
	\item Terminal Identifier
	\item Optional
	\subitem Terminal Equals
	\subitem Non-terminal Expression
\end{itemize}

Based on this, the parse declaration is implemented as seen in listing \ref{lis:parseDeclaration}.
Do note, that the final parse declaration method is a little more complex due to type inference and type checking.

\begin{lstlisting}[language=java,label=lis:parseDeclaration,caption=A simplified version of the parse declaration method from the parser.]
private fun parseDeclaration(): TreeNode.Command.Declaration {
    val type = parseType()
    val identifierToken = accept<Token.Identifier>()
    val identifier = AstNode.Command.Expression.Value.Identifier(identifierToken.value)
    val expression = if (current.token == Token.SpecialChar.Equals) {
	    accept<Token.SpecialChar.Equals>()
	    parseExpression()
    } else null
    return TreeNode.Command.Declaration(type, identifier, expression)
}
\end{lstlisting}

As seen in the listing, the method will parse the type, accept the identifier, accept the equal sign, and at last it will parse the expression.

\textbf{Deciding what to parse}

The sealed classes along with the pattern matching of kotlin are excellent tools to decide what should be parsed. 
In listing \ref{lis:parseCommand} a subset of the parse command method is shown. If the current token is a type, a declaration should be passed. If the current token is an identifier it is the next token that determines whether to parse an assignment or an expression.

\begin{lstlisting}[language=java,label=lis:parseCommand,caption=A simplified version of the parse declaration method from the parser.]
private fun parseCommand(): AstNode.Command {
    val command = when (current.token) {
	    is Token.Type -> parseDeclaration()
	    is Token.Identifier ->
		    if (peek().token == Token.SpecialChar.Equals) parseAssignment() 
		    else parseExpression()
    
    ...
    
    flushNewLine()
    return command
}
\end{lstlisting}

% we may also need to document some of the complex parts of the parser, but we will wait with that, until it is actually more or less completed.

\textbf{Type checking}

% This is where Jonas symbol table implementation part should be inserted, maybe with a little explanation of the coupling between the parser and the symbol table (through the type checker.)