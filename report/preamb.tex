\documentclass[a4paper,12pt,fleqn,oneside,openright]{memoir} 	% Openright aabner kapitler paa hoejresider (openany begge)

\makeatletter
%%%% PAKKER %%%%
\usepackage[usenames,dvipsnames]{xcolor}
% ¤¤ Oversaettelse og tegnsaetning ¤¤ %
\usepackage[utf8]{inputenc}					% Input-indkodning af tegnsaet (UTF8)
\usepackage[english]{babel}					% Dokumentets sprog
\usepackage[T1]{fontenc}					% Output-indkodning af tegnsaet (T1)
\usepackage{ragged2e,anyfontsize}			% Justering af elementer

\usepackage[bottom]{footmisc}

% TIKZ %
\usepackage{pgf}
\usepackage{tikz}
\usepackage{mathrsfs}

% Math %
\usepackage{amsmath}
% Cartasian product %
\newcommand{\Cross}{\mathbin{\tikz [x=1.4ex,y=1.4ex,line width=.2ex] \draw (0,0) -- (1,1) (0,1) -- (1,0);}}

%Checkmark
\def\checkmark{\tikz\fill[scale=0.4](0,.35) -- (.25,0) -- (1,.7) -- (.25,.15) -- cycle;}

%tabularx
\usepackage{tabularx,ragged2e}
\newcolumntype{L}{>{\centering\arraybackslash}m{1.75cm}}
\newcolumntype{C}{>{\arraybackslash}m{2cm}}

%Antal sider
\usepackage{totpages}

%For appendix
\usepackage{appendix}
% ¤¤ Figurer og tabeller (floats) ¤¤ %
\usepackage{graphicx} 						% Haandtering af eksterne billeder (JPG, PNG, PDF)
\usepackage{multirow}                		% Fletning af raekker og kolonner (\multicolumn og \multirow)
\usepackage{colortbl} 						% Farver i tabeller (fx \columncolor, \rowcolor og \cellcolor)
%\usepackage{xcolor}				            % Definer farver med \definecolor. Se mere: http://en.wikibooks.org/wiki/LaTeX/Colors
\usepackage{flafter}						% Soerger for at floats ikke optraeder i teksten foer deres reference
\let\newfloat\relax 						% Justering mellem float-pakken og memoir
\usepackage{float}							% Muliggoer eksakt placering af floats, f.eks. \begin{figure}[H]
\usepackage{wrapfig}						% Indsaettelse af figurer omsvoebt af tekst. \begin{wrapfigure}{Placering}{Stoerrelse}
\usepackage{subfig}

% ¤¤ Matematik mm. ¤¤
\usepackage{amsmath,amssymb,stmaryrd} 		% Avancerede matematik-udvidelser
\usepackage{mathtools}						% Andre matematik- og tegnudvidelser
\usepackage{textcomp}                 		% Symbol-udvidelser (f.eks. promille-tegn med \textperthousand )
\usepackage{siunitx}						% Flot og konsistent praesentation af tal og enheder med \si{enhed} og \SI{tal}{enhed}
\sisetup{output-decimal-marker = {,}}		% Opsaetning af \SI (DE for komma som decimalseparator)
\usepackage[version=3]{mhchem} 				% Kemi-pakke til flot og let notation af formler, f.eks. \ce{Fe2O3}

% ¤¤ Referencer og kilder ¤¤ %
\usepackage[danish]{varioref}				% Muliggoer bl.a. krydshenvisninger med sidetal (\vref)
\usepackage[numbers]{natbib}							% Udvidelse med naturvidenskabelige citationsmodeller
\usepackage[colorlinks]{hyperref}
\setcitestyle{number}

% ¤¤ Misc. ¤¤ %
\usepackage{listings}						% Placer kildekode i dokumentet med \begin{lstlisting}...\end{lstlisting}
% Kode-style
\lstdefinestyle{Kotlin}{language=kotlin, frame=lr, rulecolor=\color{blue!80!black}}
\renewcommand{\lstlistingname}{Snippet}                                            
\lstdefinelanguage{Kotlin}{
	keywords={package, as, as?, typealias, this, super, val, var, fun, for, null, true, false, is, in, throw, return, break, continue, object, if, try, else, while, do, when, class, interface, enum, object, companion, override, public, private, get, set, import, abstract, vararg, expect, actual, where, suspend, data, internal, dynamic, final, by},
	keywordstyle=\color{NavyBlue}\bfseries,
	ndkeywords={@Deprecated, @JvmName, @JvmStatic, @JvmOverloads, @JvmField, @JvmSynthetic, Iterable, Int, Long, Integer, Short, Byte, Float, Double, String, Runnable, Array},
	ndkeywordstyle=\color{BurntOrange}\bfseries,
	emph={println, return@, forEach, map, mapNotNull, first, filter, firstOrNull, lazy, delegate},
	emphstyle={\color{OrangeRed}},
	identifierstyle=\color{black},
	sensitive=true,
	commentstyle=\color{gray}\ttfamily,
	comment=[l]{//},
	morecomment=[s]{/*}{*/},
	stringstyle=\color{ForestGreen}\ttfamily,
	morestring=[b]",
	morestring=[s]{"""*}{*"""},
}

\lstdefinestyle{HCL}{language=hcl, frame, lr, rulecolor=\color{blue!80!black}}
\renewcommand{\lstlistingname}{Snippet}
\lstdefinelanguage{HCL}
{
	keywords={true, false, return, return}, 
	keywordstyle=\color{ForestGreen}\bfseries,
	ndkeywords={none, var, txt, num, bool, tuple, list, func, value},
	ndkeywordstyle=\color{BurntOrange}\bfseries,
	emph={map, filter, reduce, print, add, where, do, equals, thenElse}, %add all included functions here
	emphstyle={\color{RoyalBlue}},
	identifierstyle=\color{Maroon},
	sensitive=true,
	commentstyle=\color{Gray}\ttfamily,
	comment=[l]{\#},
	stringstyle=\color{black}\ttfamily,
	morestring=[b]",	
}
                                       
                                       
\lstdefinestyle{CSharp}{language=[sharp]c, frame=lr, rulecolor=\color{blue!80!black}}

\renewcommand{\lstlistingname}{Snippet}     % Listing navn


%Words
\lstdefinelanguage{CSharp}
{
sensitive=true,
morekeywords=[1]{
abstract, as, base, break, case,
catch, checked, class, const, continue,
default, delegate, do, else, enum,
event, explicit, extern, false,
finally, fixed, for, foreach, goto, if,
implicit, in, interface, internal, is,
lock, namespace, new, null, operator,
out, override, params, private,
protected, public, readonly, ref,
return, sealed, sizeof, stackalloc,
static, struct, switch, this, throw,
true, try, typeof, unchecked, unsafe,
using, virtual, volatile, while, bool,
byte, char, decimal, double, float,
int, lock, object, sbyte, short, string,
uint, ulong, ushort, void,
},
morekeywords=[2]{},
morekeywords=[3]{dictionary, Dictionary, StopWordTool, Math, math, List, list, KeyValuePair},
morekeywords=[4]{},
morecomment=[l]{//},
morecomment=[s]{/*}{*/},
morecomment=[l][keywordstyle4]{\#},
morestring=[b]",
morestring=[b]',
}

%Coloring
\lstset{
backgroundcolor=\color[rgb]{0.95, 0.95, 0.95},
tabsize=2,
numbers=left,
rulecolor=,
numberblanklines=false,
basicstyle=\scriptsize,
upquote=true,
aboveskip={1.5\baselineskip},
columns=fixed,
showstringspaces=false,
extendedchars=true,
breaklines=true,
escapeinside=||,
prebreak = \raisebox{0ex}[0ex][0ex]{\ensuremath{\hookleftarrow}},
frame=single,
showtabs=false,
showspaces=false,
showstringspaces=false,
identifierstyle=\ttfamily,
keywordstyle=\color[rgb]{1.0,0,0},
keywordstyle=[1]\color[rgb]{0,0,0.75},          %Blue
keywordstyle=[2]\color[rgb]{0.5,0.0,0.0},       %Red
keywordstyle=[3]\color[rgb]{0.127,0.427,0.514}, %Turquoise
keywordstyle=[4]\color[rgb]{0.4,0.4,0.4},       %Grey
commentstyle=\color[rgb]{0.133,0.545,0.133},    %Dark green
stringstyle=\color[rgb]{0.639,0.082,0.082}, 
}
\let\origthelstnumber\thelstnumber
\belowcaptionskip=-10pt

\newcommand*\Suppressnumber{%
  \lst@AddToHook{OnNewLine}{%
    \let\thelstnumber\relax%
     \advance\c@lstnumber-\@ne\relax%
    }%
}

\newcommand*\Reactivatenumber[1]{%
  \setcounter{lstnumber}{\numexpr#1-1\relax}
  \lst@AddToHook{OnNewLine}{%
   \let\thelstnumber\origthelstnumber%
   \refstepcounter{lstnumber}
  }%
}

\usepackage{lipsum}							% Dummy text \lipsum[..]
\usepackage{enumitem}
%\usepackage[shortlabels]{enumitem}			% Muliggoer enkelt konfiguration af lister
\usepackage{pdfpages}						% Goer det muligt at inkludere pdf-dokumenter med kommandoen \includepdf[pages={x-y}]{fil.pdf}
\usepackage{longtable}                      % Goer det muligt at have longtables
\pdfoptionpdfminorversion=6					% Muliggoer inkludering af pdf dokumenter, af version 1.6 og hoejere
\pretolerance=2500 							% Justering af afstand mellem ord (hoejt tal, mindre orddeling og mere luft mellem ord)

% Kommentarer og rettelser med \fxnote. Med 'final' i stedet for 'draft' udloeser hver note en error i den faerdige rapport.
\usepackage[footnote,draft,danish,silent,nomargin]{fixme}


%%%% BRUGERDEFINEREDE INDSTILLINGER %%%%

% ¤¤ Marginer ¤¤ %
\setlrmarginsandblock{3.5cm}{2.5cm}{*}		% \setlrmarginsandblock{Indbinding}{Kant}{Ratio}
\setulmarginsandblock{2.5cm}{3.0cm}{*}		% \setulmarginsandblock{Top}{Bund}{Ratio}
\checkandfixthelayout 						% Oversaetter vaerdier til brug for andre pakker
\raggedbottom                               % Goer at sektionerne ikke fylder op fra bunden.
%	¤¤ Afsnitsformatering ¤¤ %
\setlength{\parindent}{0mm}           		% Stoerrelse af indryk
\setlength{\parskip}{3mm}          			% Afstand mellem afsnit ved brug af double Enter
\linespread{1,1}							% Linie afstand

% ¤¤ Litteraturlisten ¤¤ %
%\bibpunct[,]{[}{]}{;}{a}{,}{,} 				% Definerer de 6 parametre ved Harvard henvisning (bl.a. parantestype og seperatortegn)

%\bibliographystyle{bibtex/harvard}			% Udseende af litteraturlisten.
\bibliographystyle{bibtex/IEEEtran}			% Udseende af litteraturlisten.

% ¤¤ Indholdsfortegnelse ¤¤ %
\setsecnumdepth{subsection}		 			% Dybden af nummerede overkrifter (part/chapter/section/subsection)
\maxsecnumdepth{subsection}					% Dokumentklassens graense for nummereringsdybde
\settocdepth{subsection} 					% Dybden af indholdsfortegnelsen

% ¤¤ Lister ¤¤ %
\setlist{
  topsep=0pt,								% Vertikal afstand mellem tekst og listen
  itemsep=-1ex,								% Vertikal afstand mellem items
}


\newcommand{\pair}[2]{\textbf{#1}: \text{#2}\\}

% ¤¤ Visuelle referencer ¤¤ %
\usepackage[colorlinks]{hyperref}			% Danner klikbare referencer (hyperlinks) i dokumentet.
\hypersetup{colorlinks = true,				% Opsaetning af farvede hyperlinks (interne links, citeringer og URL)
    linkcolor = black,
    citecolor = black,
    urlcolor = black
}

% ¤¤ Opsaetning af figur- og tabeltekst ¤¤ %
\captionnamefont{\small\bfseries\itshape}	% Opsaetning af tekstdelen ('Figur' eller 'Tabel')
\captiontitlefont{\small}					% Opsaetning af nummerering
\captiondelim{. }							% Seperator mellem nummerering og figurtekst
\hangcaption								% Venstrejusterer flere-liniers figurtekst under hinanden
%\captionwidth{\linewidth}					% Bredden af figurteksten
\setlength{\belowcaptionskip}{0pt}			% Afstand under figurteksten

% ¤¤ Opsaetning af listings ¤¤ %
\definecolor{commentGreen}{RGB}{34,139,24}
\definecolor{stringPurple}{RGB}{208,76,239}

\lstset{language=Matlab,					% Sprog
	basicstyle=\ttfamily\scriptsize,		% Opsaetning af teksten
	keywords={for,if,while,else,elseif,		% Noegleord at fremhaeve
			  end,break,return,case,
			  switch,function},
	keywordstyle=\color{blue},				% Opsaetning af noegleord
	commentstyle=\color{commentGreen},		% Opsaetning af kommentarer
	stringstyle=\color{stringPurple},		% Opsaetning af strenge
	showstringspaces=false,					% Mellemrum i strenge enten vist eller blanke
	numbers=left, numberstyle=\tiny,		% Linjenumre
	extendedchars=true, 					% Tillader specielle karakterer
	columns=flexible,						% Kolonnejustering
	breaklines, breakatwhitespace=true,		% Bryd lange linjer
}

% ¤¤ Navngivning ¤¤ %
\addto\captionsdanish{
	\renewcommand\appendixname{Appendiks}
	\renewcommand\contentsname{Indholdsfortegnelse}
	\renewcommand\appendixpagename{Appendiks}
	\renewcommand\appendixtocname{Appendiks}
	\renewcommand\cftchaptername{\chaptername~}				% Skriver "Kapitel" foran kapitlerne i indholdsfortegnelsen
	\renewcommand\cftappendixname{\appendixname~}			% Skriver "Appendiks" foran appendiks i indholdsfortegnelsen
}

% ¤¤ Kapiteludssende ¤¤ %
\definecolor{numbercolor}{gray}{0.7}		% Definerer en farve til brug til kapiteludseende
\newif\ifchapternonum

\makechapterstyle{jenor}{					% Definerer kapiteludseende frem til ...
  \renewcommand\beforechapskip{0pt}
  \renewcommand\printchaptername{}
  \renewcommand\printchapternum{}
  \renewcommand\printchapternonum{\chapternonumtrue}
  \renewcommand\chaptitlefont{\fontfamily{pbk}\fontseries{db}\fontshape{n}\fontsize{25}{35}\selectfont\raggedleft}
  \renewcommand\chapnumfont{\fontfamily{pbk}\fontseries{m}\fontshape{n}\fontsize{1in}{0in}\selectfont\color{numbercolor}}
  \renewcommand\printchaptertitle[1]{%
    \noindent
    \ifchapternonum
    \begin{tabularx}{\textwidth}{X}
    {\let\\\newline\chaptitlefont ##1\par}
    \end{tabularx}
    \par\vskip-2.5mm\hrule
    \else
    \begin{tabularx}{\textwidth}{Xl}
    {\parbox[b]{\linewidth}{\chaptitlefont ##1}} & \raisebox{-15pt}{\chapnumfont \thechapter}
    \end{tabularx}
    \par\vskip2mm\hrule
    \fi
  }
}											% ... her

\chapterstyle{jenor}						% Valg af kapiteludseende - Google 'memoir chapter styles' for alternativer

% ¤¤ Sidehoved/sidefod ¤¤ %

\makepagestyle{Uni}							% Definerer sidehoved og sidefod udseende frem til ...
\makepsmarks{Uni}{%
	\createmark{chapter}{left}{shownumber}{}{. \ }
	\createmark{section}{right}{shownumber}{}{. \ }
	\createplainmark{toc}{both}{\contentsname}
	\createplainmark{lof}{both}{\listfigurename}
	\createplainmark{lot}{both}{\listtablename}
	\createplainmark{bib}{both}{\bibname}
	\createplainmark{index}{both}{\indexname}
	\createplainmark{glossary}{both}{\glossaryname}
}
\nouppercaseheads											% Ingen Caps oenskes

\makeevenhead{Uni}{Group ds310e17}{}{\leftmark}			    % Lige siders sidehoved (\makeevenhead{Navn}{Venstre}{Center}{Hoejre})
\makeoddhead{Uni}{\rightmark}{}{Aalborg University}			%Ulige siders sidehoved (\makeoddhead{Navn}{Venstre}{Center}{Hoejre})
\makeevenfoot{Uni}{}{}{\thepage}							% Lige siders sidefod (\makeevenfoot{Navn}{Venstre}{Center}{Hoejre})
\makeoddfoot{Uni}{}{}{\thepage}								% Ulige siders sidefod (\makeoddfoot{Navn}{Venstre}{Center}{Hoejre})
\makeheadrule{Uni}{\textwidth}{0.5pt}						% Tilfoejer en streg under sidehovedets indhold
\makefootrule{Uni}{\textwidth}{0.5pt}{1mm}					% Tilfoejer en streg under sidefodens indhold

%\copypagestyle{Unichap}{Uni}								% Sidehoved defineres som blank på kapitelsider
%\makeoddhead{Unichap}{}{}{}
%\makeevenhead{Unichap}{}{}{}
%\makeheadrule{Unichap}{\textwidth}{0pt}
%\aliaspagestyle{chapter}{Unichap}							% Den ny style vaelges til at gaelde for chapters
															% ... her

\pagestyle{Uni}												% Valg af sidehoved og sidefod (benyt "plain" for ingen sidehoved/fod)

%%%% EGNE KOMMANDOER %%%%

% ¤¤ Billede hack ¤¤ %										% Indsaet figurer nemt med \figur{Stoerrelse}{Fil}{Figurtekst}{Label}
\newcommand{\figur}[4]{
		\begin{figure}[H] \centering
			\includegraphics[width=#1\textwidth]{#2}
			\caption{#3}
			\label{#4}
		\end{figure}
}

% ¤¤ Specielle tegn ¤¤ %
\newcommand{\decC}{^{\circ}\text{C}}
\newcommand{\dec}{^{\circ}}
\newcommand{\m}{\cdot}

\usepackage{textcomp}
%%%% ORDDELING %%%%

\hyphenation{In-te-res-se e-le-ment}

\graphicspath{{images/}}

\DeclareUnicodeCharacter{FFFD}{YOLO}

\newtheorem{theorem}{Theorem}


\usepackage{caption,booktabs,array}
\captionsetup{labelfont=bf}
\captionsetup{format=hang}

\newcommand{\rowgroup}[1]{\hspace{-1em}#1}

%To make blank page
\def\cleardoublepage{\clearpage\if@twoside%
    \ifodd\c@page\else
    \vspace*{\fill}
    \hfill
    \begin{center}
    This page is intentionally left (almost) blank.
    \end{center}
    \vspace{\fill}
    \thispagestyle{empty}
    \newpage
    \if@twocolumn\hbox{}\newpage\fi\fi\fi
}
\makeatother

%TXT input + mulighed for flerspaltet
\usepackage{verbatim}
\usepackage{multicol}
\usepackage{tombstone}
