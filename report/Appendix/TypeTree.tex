% !TeX root = ../main.tex
\section{Type Rule Derivation Tree Example}
\label{sec:typeRuleTree}
This is an example of how to create a type derivation tree for a program written in HCL. It is based on the following code snippet.

\begin{lstlisting}[language=HCL]
num x = 5
bool b = x greaterThan 0
b print
\end{lstlisting}

The program calls two functions, $greaterThan$ and $print$.
They are declared with the following types:

\begin{lstlisting}[language=HCL]
func[num, num, bool] greaterThan # Two number parameters, returns bool
func[bool, none] print # overloaded to take a bool parameter, returns nothing
\end{lstlisting}

Following the type rules specified in section \ref{typerules}, this derivation tree will be produced:
\begin{center}
	\begin{math}
		\cfrac
		{\cfrac
			{E\vdash 5 : num}
			{E\vdash num\ x = 5 : \texttt{ok}}\quad \cfrac
				{\cfrac
					{\cfrac
						{E(x) : num\quad E \vdash 0 : num}
						{E\vdash x\ greaterThan\ 0: bool}}
					{E\vdash bool\ b = x\ greaterThan\ 0 : \texttt{ok}}\quad \cfrac
					{E\vdash b : bool}
					{E\vdash b\ print : (none, \texttt{ok})}}
				{E\vdash bool\ b = x\ greaterThan\ 0;\ b\ print : \texttt{ok}}}
		{E \vdash num\ x = 5;\ bool\ b = x\ greaterThan\ 0;\ b\ print : \texttt{ok}}
	\end{math}
\end{center}