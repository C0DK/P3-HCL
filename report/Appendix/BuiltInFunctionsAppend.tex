\section{HCL Built-In Functions}
\label{builtinAppend}
\textbf{Arithmetic Operations}\\
HCL includes five standard arithmetic functions for use on numerical values.
The functions are seen in table \ref{tbl:arith}.

\begin{table}[h]
	\centering
	\caption{Arithmetic Functions}
	\label{tbl:arith}
	\begin{tabular}{|c|c|c|c|}
		\hline
		Operation      & Symbol & Input Parameters & Return Type \\ \hline
		Addition       & +      & num, num         & num         \\ \hline
		Subtraction    & -      & num, num         & num         \\ \hline
		Division       & /      & num, num         & num         \\ \hline
		Multiplication & *      & num, num         & num         \\ \hline
		Modulo         & mod    & num, num         & num         \\ \hline
	\end{tabular}
\end{table}

\textbf{Boolean Operations}\\
Functions for boolean operations are listed in table \ref{tbl:bool}.
These functions compare numeral, textual or boolean values and returns a boolean value.

\begin{table}[h!]
	\centering
	\caption{Boolean Functions}
	\label{tbl:bool}
	\begin{tabular}{|c|c|c|c|}
		\hline
		Operation                                                       & Symbol         & Input Parameters & Return Type \\ \hline
		Smaller Than                                                    & \textless{}    & num, num         & bool        \\ \hline
		Greater Than                                                    & \textgreater{} & num, num         & bool        \\ \hline
		Numerical Equals                                                & equals         & num, num         & bool        \\ \hline
		\begin{tabular}[c]{@{}c@{}}Numerical Not \\ Equals\end{tabular} & notEquals      & num, num         & bool        \\ \hline
		And                                                             & and            & bool, bool       & bool        \\ \hline
		Or                                                              & or             & bool, bool       & bool        \\ \hline
		Textual Equals                                                  & equals         & txt, txt         & bool        \\ \hline
		\begin{tabular}[c]{@{}c@{}}Textual Not \\ Equals\end{tabular}   & notEquals      & txt, txt         & bool        \\ \hline
		Not                                                             & not            & bool             & bool        \\ \hline
	\end{tabular}
\end{table}

\textbf{Text Manipulation}\\
Functions used to manipulate text-objects or converts values to text, seen in table \ref{tbl:text}.
\begin{table}[h!]
	\centering
	\caption{Textual Functions}
	\label{tbl:text}
	\begin{tabular}{|c|c|c|c|}
		\hline
		Operation       & Symbol & Input Parameters & Return Type \\ \hline
		Concat Text     & +      & txt, txt         & txt         \\ \hline
		Numeric To Text & toText & num              & txt         \\ \hline
		Boolean To Text & toText & bool             & txt         \\ \hline
		Textual To Text & toText & txt              & txt         \\ \hline
	\end{tabular}
\end{table}

\textbf{Control Structures}\\
Functions used as control structures in HCL, seen in table \ref{tbl:control}
\begin{table}[h]
	\centering
	\caption{Control Structures}
	\label{tbl:control}
	\begin{tabular}{|c|c|c|c|}
		\hline
		Operation                                                 & Symbol & Input Parameters            & Return Type \\ \hline
		\begin{tabular}[c]{@{}c@{}}Then \\ Statement\end{tabular} & then   & bool, func{[}none{]}        & bool        \\ \hline
		While Loop                                                & while  & func{[}none{]}, bool        & bool        \\ \hline
		Foreach Loop                                              & each   & List{[}T{]}, func{[}none{]} & none        \\ \hline
	\end{tabular}
\end{table}

\textbf{List Operations}\\
Functions used to manipulate or monitor lists, seen in table \ref{tbl:list}.
\begin{table}[h]
	\centering
	\caption{List Operations}
	\label{tbl:list}
	\begin{tabular}{|c|c|c|c|}
		\hline
		Operation     & Symbol  & Input Parameters            & Return Type \\ \hline
		Get Lenght    & lenght  & list{[}T{]}                 & num         \\ \hline
		At Index      & at      & list{[}T{]}, num            & T           \\ \hline
		Concatenation & +       & list{[}T{]}, List{[}T{]}    & list{[}T{]} \\ \hline
		Get Sublist   & subList & list{[}T{]}, num, num       & list{[}T{]} \\ \hline
		Map           & map     & list{[}T{]}, func[T, T]     & list{[}T{]} \\ \hline
		Filter List   & filter  & list{[}T{]}, func[T, bool]  & list{[}T{]} \\ \hline
	\end{tabular}
\end{table}

\textbf{Pin-Functions}\\
Functions used to read and write from arduino's analogue and digital pins, seen in table \ref{tbl:pins}.
\begin{table}[h]
	\centering
	\caption{Pin Operations}
	\label{tbl:pins}
	\begin{tabular}{|c|c|c|c|}
		\hline
		Operation                                                      & Symbol      & Input Parameters & Return Type \\ \hline
		\begin{tabular}[c]{@{}c@{}}Set Digital Pin\\ HIGH\end{tabular} & HIGH        & num              & none        \\ \hline
		\begin{tabular}[c]{@{}c@{}}Set Digital Pin \\ LOW\end{tabular} & LOW         & num              & none        \\ \hline
		Read Digital Pin                                               & readDigital & num              & num         \\ \hline
		Write Analog Pin                                               & writeAnalog & num, num         & none        \\ \hline
		Read Analog Pin                                                & readAnalog  & num              & num         \\ \hline
		Delay                                                          & delayMillis & num              & none        \\ \hline
	\end{tabular}
\end{table}

\textbf{Print Operations}\\
Functions used to print to output, seen in table \ref{tbl:print}.
\begin{table}[h]
	\centering
	\caption{Print Functions}
	\label{tbl:print}
	\begin{tabular}{|c|c|c|c|}
		\hline
		Operation  & Symbol    & Input Parameters & Return Type \\ \hline
		Print      & print     & num/bool/txt     & none        \\ \hline
		Print Line & printLine & num/bool/txt     & none        \\ \hline
	\end{tabular}
\end{table}