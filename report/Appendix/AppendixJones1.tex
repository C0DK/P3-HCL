\section{Aalborg Tekniske Gymnasium Student and Teacher Interview (In danish)}
\label{Interviews}
Projektstyring:
Analog og digital teknik.
Programmerbar teknologi.
Næsten ingen programmering i faget.
Allerede har haft programmering: Visual basic og C++
- B-niveau: Variabler, typer, kontrolstrukturer. 


Projekter:
- Cykellygte,
- Spilprojekt på arduino,
- Motor (Robot),
- Censorprojektet,
- Eksamensprojekt.

Spørgsmål: 
- Lave en funktion (opbygning)
- Lang tid på fejlsøgning.

Snakker meget for visuel programmering.
Giver forståelse for de initielle aspekter.


- Semikolon: Måske - (Men ja)
- Typer: Brug begge dele.
- Static vs dynamic typed: Så længe det virker. 
(Fare for at gøre noget uhensigtsmæssigt)

Faget: Ikke så meget forstå - mere bare lave.
 - Mindre fysik.

Er kurset i høj kurs: Har haft programmering -> vælger dette fag.

Digital design og udvikling (nyt fag, apps, spil, robotdesign)

Nogle skifter hurtigt. 
Andre har ingen erfaring, men vokser med opgaven.

Generelle hjælpelinjer:
- Nem tilkobling og interaktion med hw.

GPIO: LED - DC Motor. 
Nogle sensorer. 
Bluetooth.

Duden er hw fan.

=================================

Til elever:

Lidt C{\#} -> Programmering (C{\#})
Meget C{\#}
Godt lide computere, men programmering for "boxed" - Ikke skrive forkert. 
"Hyggeligt når det fungere"
Slet ingen programmering.
Opnået en del interesse igennem faget.
Linjefag. 
Alle har haft C{\#}.
Linjefag. 
C{\#} Programmering.
En rimelige skarp i javascript.

Programmering:
- Rigtige funtkionskald,
- Semikolon er fint - Det er okay (MEN PISSE TRÆLS NÅR MAN GLEMMER)
- Arrays - 0-indexed forvirrende - arrays forvirrende.
- Var ikke brugt. 
-> auto brugt.
- Kender ikke funktionskald (Mange forskellige)
- Problemer med typer - (Type inference ville være nemmere) - Men kan godt lide eksplicit type.
- Semikolon ikke træls - Skulle være.
- Aner intet om typer 
- Ikke problemer med typer.
- C{\#} Let at gå til -> IDE'en. 
- C vs C{\#} i ide -> De mener at det ville være lige svært i en almindelig text editor.
- Loops var svære. 
Forskel på loops.
- Kunne ikke se pointe med metoder.
- Nemmere uden typer. 
Lidt svært at lære typer.
- Svært ikke at lære typer til start?
- Synes det var fint ikke at forholde sig til typer i start (brugte lua)
- Meget udenadslære, ikke tænke sig til hvilke kommandoer der gør hvad. 
Det kunne være bedre fra IDE'en.
- Loops er forvirrende. 
"For"-navnet er ikke intuitivt. 
- En med JS synes godt med var.
- Lang tid på at lære typerne.
- Kunne samle tal typer til num.

Hvad kunne være bedre?: 
- Dokumentationen kunne være bedre.
- Man er vant til overhead -> Så det okay, men lidt tvilende. 
Men de mener også det er fordi det er det de har lært, og det nok kunne være simplere for begyndere.
- Error beskeder lort.
- Smart med ikke include.
- Kan ikke huske hvordan man laver funktion, men de siger det altid nemt at se hvordan man gjorde sidst.
- Bedre dokumentation
- Overflod af metoder.
- Eksempler til sproget.
- Lidt problemer med header og includes.
- Scoping er lidt svært.
- Nem tilgang til muligt funktions kald.
- Godt med afgrænsning fra {}
- En forslår indentation som fra python.

Arduino platformen:
- Den er fin, IDE dårlig ikke hjælper.
- Minestorm forturkket af en der er "Dårlig" til programmering.
- Love it. 
Det er simpelt.
- Arduino er "Dum" gør hvad den bliver bedt om -> godt.
- Svært med for mange komponenter, men generelt god læringsplatform.
- Godt med hands-on experience. 
Bedre med noget fysisk, end bare en terminal fra C{\#}.
- Svært med hvad forskellige gpio og ports gør.
- Ideen med at man får noget til at "Lyse" er fed!
- Fejlbeskeder er dårlige.
- Editoren burde hjælpe mere.


Læse videre:
- Software
- Nanoteknologi
- Maskinmester
- Programmering (Presset mod det.)
- Datalog 
- Bedre sprog kunne have øget interesse.
- Maskinmester
- En ville lave hjemmesider, men synes programmering ikke var underholdende, fordi læringskurve var for stejl.
- Vil gerne hurtigere nå målet.
- Sidste gruppe snakker rigtig meget for at et nemmere sprog kunne have øget interessen, og at de er blevet skræmt lidt væk.
- En software ingeniør. 


- Vores sprog: 
- Ikke så synligt for dem. 
Men efter forklaring gav mening.

- Første 3 gik meget hurtigt og klar igennem, 4 var lidt mere besværlig. 
Hjalp med parenteser.

- Gruppe løste alle 4 opgaver.
(Denne gruppe havde IKKE programmeret før)

- Denne gruppe acede det. 
Stadig en smule problemer med associering.

- Sidste gruppe var rimelig god, men stadig lidt tvivl om associering.