% !TeX root = ../main.tex
\begin{landscape}
\section{Semantics Derivation Tree Example}
\label{sec:semanticsTree}
This is an example of how to create a derivation tree for a program written in HCL.
It is based on the following code snippet.

\begin{lstlisting}[language=HCL,label=lis:derivationCode,firstnumber=1]
num x
x = 5
\end{lstlisting}

Using the abstract syntax described in section \ref{sec:semantics}, the code can be written as the following big step semantic:

\begin{center}
	$env_V, env_F \vdash \langle num\ x;\ x = 5, env_V, env_F, sto \rangle \rightarrow_{stm} (sto', env_V', env_F')$
\end{center}
Using the Compositional Big Step Semantic rule, the following derivation is produced:

\begin{center}
	\begin{math}
		\cfrac
		{env_V, env_F \vdash \langle num\ x, env_V, env_F, sto \rangle \rightarrow_{stm} (sto'', env_V'', env_F'')\quad env_V'', env_F'' \vdash \langle x = 5, env_V'', env_F'', sto'' \rangle \rightarrow_{stm} (sto', env_V', env_F')}
		{env_V, env_F \vdash \langle num\ x;\ x = 5, env_V, env_F, sto \rangle \rightarrow_{stm} (sto', env_V', env_F')}
	\end{math}
\end{center}

Table \ref{tbl:progstates} shows how the program state changes after each statement in the code is run.

\begin{table}[H]
	\centering
	\caption{Table of program states after each statement.}
	\label{tbl:progstates}
	\setlength\extrarowheight{5pt}
	\begin{tabular}{|c|c|c|}
		\hline
		& Abstract program state      & Concrete program state                          \\ \hline
		Initial state          & $(sto, env_V, env_F)$       & $(sto, env_V, env_F)$                           \\ \hline
		After first statement  & $(sto'', env_V'', env_F'')$ & $(sto[l \mapsto 0], env_V[x \mapsto l], env_F)$ \\ \hline
		After second statement & $(sto', env_V', env_F')$    & $(sto[l \mapsto 5], env_V[x \mapsto l], env_F)$  \\ \hline
	\end{tabular}
\end{table}

\end{landscape}