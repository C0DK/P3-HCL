% point of document
% We want to explain to the reader that (and how) the arduino platform is used in HTX
% The point is:
% - What "teknikfag" is.
% - What "el-teknik" is.
% - Document out findings of the student and teacher interviews.

\section{Arduino and their usage in danish education}
During the third year of the HTX\footnote{Higher Technical Examination Program, is a 3-year vocationally oriented general upper secondary programme which builds on the 10th-11th form of the Folkeskole\cite{htx_wiki}} programme students are required to take a course in a voluntary technical subject, which utilizes a project-based learning practice. 
There are a multiple of courses, one of these regard electronics, where many schools choose to utilize the Arduino platform\footnote{based on the fact that 4 of the students in this group studied at HTX in different parts of Denmark and all utilized the Arduino platform in this course. 
The HTX institution contacted by the group also utilize the Arduino platform}\cite{holstebro_education}.0

This technically oriented course, often called electronics\footnote{The name varies between institutions}, aims to familiarize the students with electrical components and circuits as well as basic programming.\cite{holstebro_electronic}
Arduino is an ideal platform for this, as it is possible to connect I/O devices to the Arduino micro-controller, which in turn can be programmed to do simple tasks. 
This course is then used to spike the interest of the student regarding engineering fields, such as programming or electronics.


\subsection{Aalborg Tekniske Gymnasium Electronic Course}
In relation to the discipline of understanding the initial problem, the project group contacted one of the major HTX institutions in North Jutland, Denmark, Namely Aalborg Tekniske Gymnasium\footnote{Aalborg Technical Gymnasium, https://aatg.dk/}. 
The following subsection will summarize the important and noticeable aspects of the electronic course conducted on Aalborg Tekniske Gymnasium. 
The conclusions reached in this subsection are purely based on the statements made by the students and the teacher of this specific electronics course.
The notes, from where the below conclusions were derived, can be found in the Appendix 1 on page \pageref{Interviews}.

The electronics course itself does not teach the students programming directly. 
Instead it focuses on project based arduino development. 
Educational programming experience of the students (if any), originates from another optional course called "Programming". 
The Programming course teaches the students the basic constructions and terminology of general programming. 
The students are taught C{\#} in the programming course. 

The electronics course focuses on engaging the students in making arudino projects, not teaching the students the theory behind the projects. 
This means that the students both have a very limited understanding of the arduino platform's architecture and a low understanding of the arduino language as explained earlier in the report, which they utilize in the electronics course on Aalborg Tekniske Gymnasium.

Because of all this the electronics course has a relatively high learning curve. 
Which results in a high number of students abandoning the course in favor of some of the courses with a more defined structures.

Clearly there is a major gap between the students who are interested in programming or electronics and have no prior experience and students who have prior experiences. 
Based on students' and the teacher's, statements it seems like the course need a better gateway into the arduino platform, the programming aspects of the course or both.

A substantial number of the students in the class expressed their frustration with the relative complexity of the arduino language. 
The teacher supplemented with saying that the language is not necessary complex, but the time spent on the different concepts is far too short, which is a result of the education being a junior education and not a college/university level education.
Clearly there is a need for the course to be able to make shortcuts in some aspects to teach the students the concepts adequately. 
Simply put the language and the platform is not well suited for this kind of low level education.

\subsubsection{The Students Understanding of Syntax and Semantics}
Most of the students expressed that they have no real interest in the syntax. 
In fact a substantial part indicated that they did not understand what the syntax of a programming language actually means. 
Below is a short summation of the most problematic aspects of general syntax and semantics of the arduino language (or any reference language the students might have), in arbitrary order.

\textbf{Data Types}\\
Data types in general, seem to cause problems for the students. 
Most students can name the different types but far less have the proper knowledge to identify and explain the meaning behind them. 
The students in most cases do not understand why they have to state the data type explicitly, when declaring a variable. 
When presented with a small example program with type inference on all declarations, the students in most cases, instantly understood how to identify a declaration and how to use the variable properly.

\textbf{Function Call and Definition}\\
The students concluded that the manner in which C++ and C{\#} makes function/method calls and definitions is somewhat ambiguous to them. 
In relation to syntax, they expressed some minor frustrations concerning the overhead necessary to define simple methods, such as methods that does not return a value and also when making function calls with no arguments. 
The most problematic aspect of abstraction in general was the matter of scoping. 
No students expressed any real knowledge of scoping. 
This is especially a problem when it comes to understanding error messages from the compiler. 
Clearly scoping rules must be kept in mind when developing a new language for beginners.

\textbf{Control Structures}\\
In general the students understood what the different control-structures do. 
But almost non of them could remember the specific syntax and meaning behind them. 
The selective control structures seems to be easier to understand than the iterative control structures. 
A substantial amount of the students stated that the control structures in general are difficult to understand and remember, since they do not reflect anything else in the language, at least according to them.
Clearly the students needs more consistency in the language. 
The control structures should in some way mirror the syntax of another major aspect of the language, to make it more intuitive to use and remember.

\textbf{Operator Precedence and Association}\\
Both terms were completely foreign to the students. 
Their perception of precedence were mostly based on arithmetic notation. 
This is sufficient in most cases. 
The biggest concern was with association. 
Most of students did not understand that different operators could have different association, which proved to be a major issue when presented with code. 
In most cases the students understood which part of the expression was calculated first. 
But chose the wrong answer based on mostly random factors. 
Clearly association needs to be as intuitive as possible.

\subsection{Summary}
Based on interviews done with students having this course, the language used for Arduino, C++, has a steep learning curve, considering the experience of the students. 
The Arduino, from a hardware point of view, is quite intuitive and it is easy to play around with lights, I/O, modules and components that give the student a visual feedback. 
Even though the students have no actual understanding of the architecture of the platform they essentially do not need it. 
Many students interviewed felt like the programming aspects of the course are far too difficult to understand in its entirety (which is reasonable), but even simple syntactic understanding of the programming language are difficult to understand for most students. 



