% This section is meant to introduce high-order functions, so we can refer to this section in the implementation part of the report.
\section{High-order functions}
High-order functions is a concept that has been discussed and valued as an important key feature of HCL.
This section is only meant to introduce high-order functions to the reader.
High-order functions allow for some powerful elements to be added to the language.
Along with this, high-order functions may also drastically improve development time, since it allows for more generic functions that can be specialized through high-order functionality.

\subsection{Explanation of high-order functions}
High-order functions either returns a function or have a function as one of its parameters. 
This language feature is often utilized together with the three functions map, filter and reduce. 
To demonstrate, these are shown in snippet \ref{lis:hofExamples}. 
Note that the function parameters in the example are passed as lambda expressions.
Lambda functions are anonymous inline functions, which are useful when used along with high-order functions, since the functionality can be specified, without having to declare an entire function.

\begin{lstlisting}[language=Python,label=lis:hofExamples,caption=An example of the map{,} filter and reduce functions.,firstnumber=1]
myList = [5, 10, 15, 20, 25]

myNewNum = myList.filter{ num => num % 10 == 0 }.map{ num => num * x }.reduce{ sumOfNums, num => sumOfNums + num }

print(myNewNum) // 500
\end{lstlisting}

In the example, a list, \texttt{myList}, with the numbers 5, 10, 15, 20 and 25 is initialized. 
The 3 functions are chained to get the sum of all these numbers squared, where the initial number was a divisor of 10.
The first function used is the \texttt{filter} function. 
It takes a list along with a predicate and returns a new list with all the elements that matched the predicate.
In the example, the predicate is whether the number is a divisor of 10.

Next is the \texttt{map} function. 
This function takes a list of numbers along with a modify function. 
It will apply the modify function on all the elements in the list, and return a list with all the modified elements.
In the example, the modify function will simply square the input number. 

The last function is the \texttt{reduce} function (often called the fold function).
This function receives a list and a fold function as input. 
The fold function will have an accumulator and the current element as parameters.
In the example, the numbers in the list are accumulated.

\subsection{Benefits of high-order functions}
The major benefit of high-order functions, is the ability to create abstract functions that will have their behavior defined through the input function. 
The \texttt{map}, \texttt{filter}, and \texttt{reduce} functions are by themselves useful, but they are also often used to create specialized functions. 
For instance a \texttt{sum by} function, which is a combination of the \texttt{reduce} and \texttt{map} function. 

Another common example is sorting. 
A generic sorting algorithm can be implemented, and a comparator function can then be passed to the sorting algorithm to tell it how to compare the different elements.
An example could be when comparing \texttt{person} objects.
The developer may want to sort the persons in alphabetical order, based upon their age, or something else. 
The only thing that the sorting function needs is a function that can compare the persons that are to be sorted.

\subsection{High-order functions in HCL}\label{subsec:hofhcl}
Although some of the use cases of high-order functions may be too complex for new programmers, the high-order functions are still valued as an important feature.
This is due to several reasons.

First of all, new programmers will often tend to do some basic mathematical computations. 
The high-order functions like map, reduce and filter will make these computations a lot easier when dealing with collections. 

Secondly, high-order functions are a useful programming concept that HCL should strive to make intuitive and understandable for new programmers, so that they will be able to use and understand it in their next language.
