\section{High-order functions}

High-order functions is a concept that has been discussed and valued as an important key feature of HCL.
High-order functions allow for some powerful elements to be added to the language.
Along with this, high-order functions may also drastically improve development time, since it allows for more generic functions that can be concretized through the high-order functions.

\subsection{Explanation of high-order functions}
High-order functions are functions that either returns a function or takes a functions as one of its parameters. 
This language feature is often utilized together with the three functions map, filter and reduce. 
To understand what these does, and why they are useful have a look at the example pseudo code in \ref{lis:hofExamples}. 
One thing to note, is that the function parameters in the example is passed as lambda functions. 
Lambda functions are anonymous inline functions, which are very useful along with high-order functions, since it will allow you to specify the functionality without having to declare an entire function.

\begin{lstlisting}[language=Python,label=lis:hofExamples,caption=An example of the map\, filter and reduce functions.,firstnumber=1]
myList = [5, 10, 15, 20, 25]

myNewNum = myList.filter{ num => num % 10 == 0 }.map{ num => num * x }.reduce{ sumOfNums, num => sumOfNums + num }

print(myNewNum) // 500
\end{lstlisting}

First a list with the numbers 5, 10, 15, 20 and 25 is initialized. 
The 3 functions are chained to get the sum of all these numbers squared, where the initial number was a divisor of 10.
The first function used is the filter function. 
It takes a list along with a predicate and returns a new list with all the elements that matched the predicate.
In the example the predicate is whether the number is a divisor of 10.

Next is the map function. 
This function takes a list of numbers along with a modify function. 
It will apply the modify function on all the elements in the list, and return a list with all the modified elements.
In the example the modify function will simply square the input number. 
This means that the returned list will have all the input numbers squared.

The last one is the reduce function (often also called the fold function).
This function will take a list and a fold function as input. 
The fold function will have an accumulator and the current element as function parameters.
In the example, the numbers in the list are accumulated.

\subsection{Benefits of high-order functions}
The major benefit from the high-order functions is the ability to create abstract functions that will have their behaviour defined through the input function. 
The map, filter, and reduce functions are by themselves very useful, but they are also often used to create a lot more concrete functions. 
For instance a "sum by" function, which is a combination of the reduce and map function. 

Another common example is for sorting. 
A generic sorting algorithm can be implemented, and a comparator function can then be passed to the sorting algorithm to tell it how to compare the different elements.
This could be useful; for instance when comparing a person class. 
The developer may want to sort the persons in alphabetical order, based upon their age or something else. 
The only thing that the sorting function needs is a function that can compare the persons that are to be sorted.

\subsection{High-order functions in HCL}
Although some of the use cases of high-order functions may be too complex for the new users, who are the target group of HCL, the high-order functions are still valued an important feature.
This is due to several reasons.

First of all, new users will often tend to do some basic mathematical computations. 
The high-order functions like map, reduce and filter will make these computations a lot easier when dealing with collections. 

Secondly, high-order functions are a useful programming concept that HCL should strive to make intuitive and understandable for new users, so that they will be able to use and understand it in their next language.