%This section is meant to specify the scope of project, both in terms of the report, the language that will be developed and the compiler.
\section{Scope}
As stated in the above analysis, the purpose of this project is to design, create and implement a
programming language for the Arduino platform that is simple for beginners to use, whilst still teaching the
fundamentals of programming.
In other terms, the language is meant to enable beginners to quickly and easily write
code that can be executed, without losing the functionality of a high-level language.

It is imperative to consider the needed functionality in order responsibly compose a language for
beginners.
After conducting interviews with gymnasium students who can be described as beginners at programming,
it was decided that the language for this project should be developed as an introductory language for beginners to use, before making a switch to a more fleshed out programming language.
As such the language is meant to introduce the thought process used in programming, while ensuring that the user will not have to worry too much about syntax.
%insert a reference on the next line to the ArduinoInEducation section
As was discovered during the visit to the gymnasium class (section <INSERTREFERENCE>), there were a couple of concepts that would be especially important to be wary of when developing the language.

The students seemed to have issues with syntax, data types, defining functions, control structures and operator precedence.
Therefore the developed language will need to have a heightned focus on these particular concepts. 
Adding to these requirements, the project group would also like to reduce the time spent on setup as much as possible, allowing the users of the language developed to quickly get from idea to working solution, as quickly as possible.
The goal for the language should always be simplicity, comprehensibility and ease of use.

Finally, the group fully intends to develop a full language, with a working compiler, that is production ready, meaning a programming language that will be usable in education.

\section{Problem statement}
Based on the analysis the following problem statement was defined:
\begin{center}
	How can a simpler programming language for the arduino platform, increase beginners interest and motivation for programming.
\end{center}
\begin{itemize}
	\item How can simplicity and a quick development process be ensured?
	\item How can the basic programming concepts be taught in a manner that ensures that the user will understand the concepts in more advanced languages?
\end{itemize}
%Dont know if we should add more here. If you have any ideas for different problem statements, do feel free to add them :)

 