% This section is meant to introduce a couple of languages for beginners.
% This section will be used to establish examples of good languages for introducing programming for beginners.
\section{Languages for beginners}

There are a wide variety of different tools, games and languages with the intent to teach students how to program. 
These span from very specific game creation tools, to general purpose programming languages. 
In order to gather some inspiration for the group's own programming language, some of the existing solutions, and the most common languages for beginners are analyzed.
Only a few of them are described in detail.

\subsection{Scratch}
\label{sec:Scratch}
Scratch is a programming language that let the user program interactive stories, animations and games. 
It is one of the most common programming languages among youngsters and students who are just getting into programming. 
It is designed for children from the age of 8 to 16, and aims to help children "think creatively, reason systematically, and work collaboratively". 
In Scratch the user do not actually have to write any code, since it utilizes drag and drop elements to build and structure the program. 
Although no exact programming is actually done, Scratch helps to bring an understanding of the common programming concepts like variables, loops and conditionals\cite{ScratchWebsite}.

Scratch is one of the most popular block building programming languages, but there are a lot of different languages challenging it, for instance Google's refinement of Scratch which is called Blockly\cite{Blockly}.

\subsection{GameMaker}
GameMaker is a tool for developing games. 
It is arguably more complicated than Scratch, but also brings more functionality, while still aiming to be easy to use. 
Quoting the GameMaker website: 

\textit{"Making games development accessible to everyone means taking away the barriers to getting started. 
Using our intuitive ‘Drag and Drop’ development environment you can have your game up and running in a matter of minutes without ever having to write any code!"}\cite{GameMaker}.
 
Besides using drag and drop, like Scratch, GameMaker also created a very simple programming language called "GML".
GML resembles the syntax of common languages like C and Java. 
However, it is a simple language, for instance type declaration is not needed or even supported.
One thing to note is that the type system is indeed static, but types are inferred. 
As such types may not be altered at runtime\cite{GML}.

\subsection{Python}
Although not directly aimed at students or youngsters, Python is a common entry point for many newcomers. 
Python also advertises itself as easy to learn for both new and experienced programmers. 
It provides a simple syntax, along with dynamic typing and type inference\cite{PythonWebsite}.
An example of Python's simple syntax is demonstrated in the "Hello World" program, presented below.

\begin{lstlisting}[language=Python,label=lis:PythonHelloWorld,caption=Hello World in python]
print("Hello World")
\end{lstlisting}

Now compare this to that of C++:

\begin{lstlisting}[language=C++,label=lis:C++HelloWorld,caption=Hello World in C++]
#include <iostream>

using namespace std;

int main() {
	cout << "Hello World" << endl;
	return 0;
}
\end{lstlisting}

The C++ "Hello World" program can be quite confusing for novices, as there is overhead.

Besides the easy and simple syntax, Python also includes a large collection of libraries\cite{PythonLibraries}.
This means the developer will be using less time reinventing the wheel and thus faster accomplishing the desired end goal\cite{PythonXKCD}.

\subsection{Takeaways}
Although the above three languages only makes up a small subset of the languages analyzed, they represent aspects of languages aimed at beginners.
Languages intended for beginners often requires little to no overhead. 
This is because the intent is to teach the students about programming concepts, before having to worry about syntax and formalities. 
These languages all use implicit type declarations.

Furthermore, the languages strive to have a simple, easy to learn syntax. 
Finally the language developers generally want to provide a lot of libraries. 
This enables the programmer to spend more time describing what should be done, rather than how to do it.
