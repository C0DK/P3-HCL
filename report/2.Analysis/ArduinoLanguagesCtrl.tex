\section{Existing languages}
Before doing any work on HCL, the project group decided to research how many high-level languages already exist for the Arduino platform, and what distinguishes them from each other.
Surprisingly, despite the popularity of the platform, there are not many options available.
Some available languages are:

\begin{itemize}
	\item The official Arduino language
	\item Occam-pi
	\item NanoVM
\end{itemize}

% Source: arduino.cc
\textbf{The official Arduino language} is, in fact, not its own language. % maybe cite arduino.cc/FAQ here because that is where this info is from
That is, it has not been built from scratch by the developers at Arduino.
It is a subset of the C++ language with its own libraries.
Because of this, it is compatible with the C and C++ languages.
There are a few changes made to the language, such as the absence of function prototypes, and the \texttt{setup}- \& \texttt{loop} functions replacing the \texttt{main} function.
However, the basic functionality and syntax of C and C++ are still present, meaning users are able to use both the imperative- and the object-oriented paradigms of C and C++ respectively to program the Arduino.

A typical program written in the Arduino language is set up like this:

\begin{lstlisting}[language=C++,label=lis:arduinoExample,caption=An example program written in the Arduino language.,firstnumber=1]
/*
 * Preprocessor directives
 */

const int PIN13 = 13;

void setup() {
    // Setup code
    pinMode(PIN13, OUTPUT); // Set digital pin 13 to output
}

void loop() {
    // Loop code is looped infinitely
    digitalWrite(PIN13, HIGH); // Set output for pin 13 to high
    delay(500);                // Delay for 500 milliseconds
    digitalWrite(PIN13, LOW);  // Set output for pin 13 to low
    delay(500);                // Delay for 500 milliseconds
}

\end{lstlisting}

% Source: concurrency.cc
\textbf{Occam-pi} is a language that focuses on making parallel programming easy for programmers.
It is based on Occam that first appeared in 1983.
The full documentation of Occam-pi can be read on the official website of the language, \url{concurrency.cc}, but here is a short description of some of the syntax.

Functions in Occam-pi are declared using the keyword \texttt{PROC} followed by a name and the function parameters enclosed in parentheses.
Unlike C, Occam-pi doesn't use curly-brackets to encapsulate the body.
Instead, the body is indented using a tab, similar to the Python language.
The end of the function is marked using a colon.
One of the key features of Occam-pi is the fact that it separates expressions that are evaluated sequentially and ones that are evaluated concurrently using the \texttt{SEQ} and \texttt{PAR} keywords.
Comments are prefixed with \texttt{"{-}{-}"}.

A typical program written in Occam-pi is set up like this:

\begin{lstlisting}[,label=lis:occamExample,caption=An example program written in Occam-pi,firstnumber=1]
--Include any needed modules
#INCLUDE "plumbing.module"

PROC main ()
	SEQ --Expressions are evaluated sequentially
		x := 4 + 3
		y := x * 5
	PAR --Expressions are evaluated concurrently
		blink(12, 1000) --a function from the plumbing module
		blink(13, 1000)
:
\end{lstlisting}

% Source: http://www.harbaum.org/till/nanovm/index.shtml
\textbf{NanoVM} is a virtual machine (VM) for the Atmel AVR ATMega8 CPU.
It allows programmers to write code in a subset of Java that can be run by Arduino and similar microcomputers.
The VM includes many of the features that have made the Java VM (JVM) popular such as object-oriented programming, automatic dynamic memory allocation, and garbage collection.
It comes packaged with native classes such as Object, System, and PrintStream.

The downsides of using NanoVM instead of the Arduino language is, as stated on the official website of NanoVM\footnote{http://www.harbaum.org/till/nanovm/index.shtml} that the VM's interpreter doesn't perform as well as C compiled to AVR code, and it reserves some of the RAM used for applications to run the VM.
NanoVM also has to be installed on the CPU's flash memory, which will overwrite the bootloader.

A typical program written in Java is set up like this:
\begin{lstlisting}[language=Java,label=lis:javaExample,caption=An example program written in Java,firstnumber=1]
public class HelloWorld {
	public static void main(String[] args) {
		System.out.println("Hello World!");
	}
}
\end{lstlisting}

% Source: https://code.google.com/archive/p/dk-basic/
These are, of course, not all the available languages for programming for Arduino.
The project group is also aware of a language called DK-BASIC, a subset of BASIC designed to work on Arduino, but that language is still in its alpha stage, so it won't get more than a mention here.
