\section{Educational mini computers}

Apart from the Arduino board, a lot of other mini computers have been created with educational purposes in mind.

\subsection{Raspberry Pi}
The raspberry pi is one of the most widely known single board computers. 
It is a complete arm-based computer, and may run anything arm based on it, including both linux and windows based arm operating systems. 
This first of all makes it a common entry point into the GNU/linux operating systsem stack, which is generally considered to very educational in regards to how computers work.
The default raspberry pi distribution ships with both scratch and python, allowing the user to quickly get his hands dirty with some programming.
On top of this, one of the major selling points to the raspberry pi is the GPIO (general purpose input output) pins. 
These allow the user to attach all kind of devices to the board, this includes both sensors, lights, cameras and breadboards. 
The raspberry pi allows the user to easily interact with the GPIO pins, creating a great device for both IoT, home automation and robotics.

\subsection{Lego Mindstorm}
Lego Mindstorm is a robot kit from lego. 
The intend is to built a robot out of lego, with the belonging motors and sensors which can then be controlled by the "head". 
The head is a small computer, which can control and monitor all the attached motors and sensors. 
The computer can be programmed from either a tablet or a computer using a drag and drop programming environment similar to that of scratches (ref scratch).

\subsection{Takeaway}

Controlling sensors, motors and GPIOS is awesome. Students like creating something concrete like a robot.