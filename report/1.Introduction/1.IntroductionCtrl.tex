
\chapter{Introduction}

In recent years, there has been a bigger focus on computer science, technology and programming in the danish school system.
This is in large part due to the prediction that Denmark will lack skilled programmers, cyber security specialists and other skilled computer science workers in the near future.
According to business insider, there will be a shortage of 19,000 IT professionals in Denmark by the year 2030.\cite{ITLackDK}

As a result, more students are being introduced to programming and programming languages during their duration at the
danish gymnasiums. 
The goal of this introduction is to give the students a basic idea of how the computers they use everyday function, but also to impart an interest or motivation for studying programming in their continued education.

As programming languages are ever evolving and new languages are constantly designed it becomes relevant
that students, who might be interested in studying programming, are introduced to the field in a way that
teaches them the fundamentals of programming.

One of the most common tools for introducing programming and electrical engineering in danish gymnasiums, is the Arduino development board. 
The Arduino platform consists of its own dedicated IDE and language.
The Arduino programming language is based on C/C++ and designed to both be easy to use, as well as being
more hardware-oriented.\cite{ArFAQ}
The Arduino platform maintains its simplicity by oftentimes shielding away a lot of complexity by ways
that are non-present in most programming languages.
Although this is an advantage in terms of simplicity, it might halter, or even frustrate, the student when learning other programming languages.

As such the motivation for this project is to create a new language for the Arduino platform, with a focus on promoting simplicity and the thought process that goes into programming, rather than difficult syntax and extraneous setup code.

%This project aims to create a new language for the Arduino platform that promotes simplicity and strives to teach student the fundamentals of programming.

From this the following initial problem statement is drafted:

\begin{center}
	\textit{\textbf{How can a programming language for Arduino be optimized for learning without 
			loosing the functionality of high-level languages}}
\end{center}

\section{Preliminary interviews with the client}
