% This chapter is aimed to introduce both the subject of the project aswell as our motivation for the project.
% It also aims to establish the intitial problem statement, which the analysis will be conducted based on.
\chapter{Introduction}
In recent years, there has been a bigger focus on computer science, technology, and programming in the danish school system.
This is in large part due to the prediction that Denmark will lack skilled programmers, cyber security specialists, and other skilled computer science workers in the near future.
In fact, according to business insider, there will be a shortage of 19,000 IT professionals in Denmark by the year 2030\cite{ITLackDK}.

As a result, more students are being introduced to programming and programming languages during their time at the
danish gymnasiums. 
The goal of these introductory courses is to give the students a basic idea of how computers work, but also to impart an interest or motivation for studying programming in their continued education.

Students, who might be interested in studying programming, need to be introduced to the field in a way that teaches them the fundamentals of programming, in an exciting environment.

One of the most common tools for introducing programming and electrical engineering in danish gymnasiums, is the Arduino development board. 
The Arduino platform consists of its own dedicated integrated development environment (IDE) and language.
The Arduino programming language is based on C++ and is designed to both be easy to use, as well as being more hardware-oriented than conventional programming languages\cite{ArFAQ}.
As the Arduino programming language is based on C++, a low level language, it can often confuse new programmers, who are not used to thinking like a programmer.

As such the motivation for this project is to create a new language for the Arduino platform, with a focus on promoting simplicity and teaching the thought process that goes into programming, rather than difficult syntax and extraneous overhead.

The language made will be named \textit{HCL Compiled Language}, or HCL for short.

Based on these considerations, the following initial problem statement has been drafted:

\begin{center}
	\textit{\textbf{How can a programming language for the Arduino platform be developed for learning without losing the functionality of high-level languages}}
\end{center}
