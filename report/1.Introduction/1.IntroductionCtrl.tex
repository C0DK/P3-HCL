
\chapter{Introduction}
In recent years, technology and programming have become a more widespread topic within the danish school
system.
More students are being introduced to programming and programming languages during their duration at the
danish gymnasiums. 
With this introduction students might amass an interest or gain a motivation for studying programming in 
their future education. \\
As programming languages are ever evolving and new languages are constantly designed it proves relevant
that students, who might be interested in studying programming, are introduced to the field in a way that
teaches them the fundamentals of programming.

One of the most common tools for learning programming and electrical engineering in danish gymnasiums is
the Arduino development board. The Arduino platform is consistent of its own dedicated IDE and language.
The Arduino programming language is based on C/C++ and designed to both be easy to use, as well as being
more hardware-oriented (https://www.arduino.cc/en/Main/FAQ\#toc2).
The Arduino platform maintains its simplicity by oftentimes shielding away a lot of complexity by ways
that are non-present in most programming languages.
Although this is an advantage for simplicity, it might halter the student when learning other programming
languages.

This project aims to create a new language for the Arduino platform that promotes simplicity and strives
to teach student the fundamentals of programming.

From this the following initial problem statement is drafted:

\centering\textit{\textbf{How can a programming language for Arduino be optimized for learning without 
loosing the functionality of high-level languages}}
\section{Preliminary interviews with the client}
